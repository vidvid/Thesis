در فصول گذشته روش آزمون ...

\section{معرفی مطالعه‌ی موردی }
\label{section:caseDesign}
\begin{itemize}
\item خصوصیت ۱

\item خصوصیت ۲
\end{itemize}

\subsection{زیر بخش}
 این فصل فرض بر آن است که موارد کاربرد برای سیستم مورد نظر تهیه شده و موجود است. 

\subsubsection{تولید مدل رفتاری سیستم}
در مدلی که در این مرحله تولید می‌شود، نیازی به مشخص کردن کنش‌‌هایی که باعث حرکت بین حالت‌های مختلف ماشین می‌شوند وجود ندارد. در واقع نمودار به دست آمده در این مرحله، تنها به هدف مدل‌سازی نمای سطح بالای عملکرد سیستم و مجموعه‌ی حالت‌های آن طراحی می‌ش

\subsubsection{تولید مدل رفتاری محیط}
آن‌چه که در ادامه‌ی این متن \emph{محیط عملکرد} و یا اختصاراً محیط نامیده می‌شود، به طور دقیق عبارت است از اکتورهایی که در یک مورد کاربرد با سیستم در ارتباطند. همان‌طور که پیش از این نیز گفته شد، برای حفظ هم‌خوانی ماشین‌های طراحی شد نمودار رفتار سیستم) نمی‌شود، بنابراین این مسیر هرگز اجرا نخواهد ش
\subsubsection{مشخص کردن و تعریف گونه‌های داده‌ای}
اگرچه این بخش به توصیف مدل‌سازی رفتاری سیستم و محیط اختصاص دارد، اما باید توجه کرد که تکمیل مدل‌های رفتاری نمی‌تواند کاملاً مستقل از داده‌هایی که بین اجزای مدل مبادله می‌شوند، انجام شود. همان‌طور که پیش از این  اصلی است. در چهارچوب پیشنهادی اجازه‌ی تعریف گونه‌های مستقل (یعنی گونه‌ای که از گونه‌ی دیگری گسترش نیافته است) وجو
\section{طراحی سیستم به روش ناهمگام}

\section{الگوها و سبک‌های طراحی}
\label{section:extracted_patterns}
پیش از این، نحو نمادگذاری مربوط به چهارچوب پیشنهادی این پژوهش تشریح شد. در این بخش، در مورد معنای هر یک از اجزای این نمادگذاری بحث خواهد شد. علاوه بر این، مطابقت معنایی این نمادگذاری را با مفاهیم آی‌اوکو مورد بررسی قرار داده و سپس الگوریتم جامعی برای تولید و اجرای یک‌پارچه‌ی آزمون‌هایی که با این نمادگذاری توصیف شده‌اند، ارائه خواهد شد.

\subsection{روشهای coordination }

\subsubsection{روش یک}
همان‌طور که پیش از این اشاره شد،‌ در چهارچوب پیشنهادی، توصیف‌های رفتاری چه برای سیستم و چه برای محیط در چندین نمودار حالت بیان می‌شود که هدف از آن کاهش پیچیدگی در طراحی مدل‌هاست. بنا به قرارداد، روش ترکیب این ماشین‌های حالت، روش میان‌گذاری است. به این معنی که 
\subsubsection{روش ۲}
همان‌طور که پیش از این اشاره شد، توصیف‌های رفتاری محیط نقش محدود‌کننده را در تولید موارد آزمون ایفا می‌کند. به عبارت دقیق‌



\subsection{سبک های طراحی}
سناریو‌های آزمون در چهارچوب پیشنهادی، تعیین کننده‌‌ی رو
با توجه به این تعاریف



\section{پیاده‌سازی}
\label{section:asyncImpl}
نحو و معنایی که برای توصیف چهارچوب پیشنهادی در این پژوهش مورد استفاده قرار گرفته است، در قالب یک مجموعه‌ی ابزار نیز پیاده‌سازی نیز شده 
