به عنوان جمع بندی متن حاضر، در این فصل به فهرستی از مهم‌ترین دستاوردهای این پژوهش خواهیم پرداخت. در مورد هر یک از این دستاوردها برخی نکات مهم نیز ذکر شده است. بعد از این، برخی از مهم‌ترین کاستی‌های چهارچوب ارائه شده آورده شده است. این کاستی‌ها در هر دو جنبه‌ی نظری و عملی مورد بررسی قرار گرفته‌اند. در نهایت، بر مبنای این موارد برخی جهت‌گیری‌های ممکن برای ادامه‌ی این پژوهش در آینده آورده شده است.

\section{دستاوردهای این پژوهش}
این پژوهش، چهارچوبی بدیع برای آزمون سیستم‌های نرم‌افزاری بر پایه‌ی روش‌های مبتنی بر مدل ارائه می‌کند که با استفاده از فرآورده‌هایی کاربردی که در فرآیند تولید نرم‌افزار تولید می‌شوند (مانند موارد کاربرد) و هم‌چنین دانش پیاده‌سازی نرم‌افزار برای تولید مدل‌های رفتاری و داده‌ای استفاده کرده، سپس به طور سیستماتیک به تولید خودکار موارد آزمون می‌پردازد. در تعریف این چهارچوب این امکان به وجود آمده که داده‌های آزمون نیز در مدلی هم‌خوان و در یک زبان یکسان با مدل‌های رفتاری (زبان یو‌ام‌ال)، طراحی شده و اطلاعات آن‌ها در مدل‌های رفتاری نیز مورد استفاده قرار گیرد. روش پیشنهاد شده‌ی این پژوهش، در ادامه برای تولید یک ابزار آزمون‌گر مورد استفاده قرار گرفته است. این ابزار از نظر نحوه‌ی تولید موارد آزمون، با توجه به ساختار الگوریتم \ref{algorithm:final}، به صورت \gls*{در-لحظه} عمل می‌کند. در فصل قبل نشان دادیم چگونه از این ابزار برای آزمون یک سیستم واقعی استفاده می‌شود.

در واقع چهارچوب پیشنهاد شده تلاش می‌کند تا مجموعه‌ی به هم‌پیوسته‌ای از فعالیت‌ها برای آزمون را، از اولین مراحل طراحی تا نتیجه‌گیری از مجموعه‌ی آزمون‌ها، پیشنهاد کند. در زیر برخی از مهم‌ترین دستاورد‌های هر یک از مراحل این کار آمده است:
\begin{strict_itemize}
\item برای طراحی مدل‌ها به طور مستقیم از فرآورده‌های تولید نرم‌افزار مانند مجموعه‌ی موارد کاربرد و یا استانداردهای موجود (مانند استاندارد \lr{ISO8583} در مطالعه‌ی موردی) استفاده شده است. هم‌چنین راهکارهایی برای تولید مستقیم مدل‌ها از موارد کاربرد (مثل قواعد مربوط به ایجاد یک ماشین برای هر مورد کاربرد و شکستن آن به اجزا) پیشنهاد گردیده است. به این ترتیب قدمی به سوی مدل‌سازی روش‌مند، به هدف آزمون برداشته شده و از اتکای کامل  به دانش ضمنی طراح آزمون پرهیز شده است.

\item تمامی توصیف‌ها، چه رفتاری و چه داده‌ای، در زبان یوام‌ال طراحی شده‌اند. علاوه بر تأثیرات مثبتی که استفاده از این زبان در طراحی آزمون‌گر داشت (مانند استفاده از ابزارهای طراحی قدرتمند و مفسّرها)، باید این را نیز افزود که امروزه حجم بسیار وسیعی از افرادی که در حوزه‌ی مهندسی نرم‌افزار فعالیت می‌کنند و حتی برخی از مشتریان سیستم‌های نرم‌افزاری به زبان یوام‌ال تسلط نسبی دارند و یا لااقل با آن آشنا هستند. این نکته خود باعث می‌شود که اولاً استفاده از راه‌کارهای ارائه شده در این پژوهش برای کاربران نهایی آسان شود و ثانیاً ارائه‌ی آن به مشتریان و کاربران خارجی نیز با سهولت بیشتری صورت گیرد.

\item الگوریتم آزمون مبتنی بر مدل به همراه داده که در \cite{FTW05} پیشنهاد شده بود از جهات مختلف بهبود داده شده است. یکی از این بهبودها امکان توصیف محدودیت‌های محیطی در کنار توصیف رفتار سیستم است. علاوه بر آن، امکان ایجاد توصیف‌های رفتاری در چند ماشین (چه در مورد توصیف سیستم و چه در مورد رفتار محیط) فراهم آمده است. اگرچه هر دوی این موارد در آزمون‌گرهای دیگری (مانند \lr{Uppaal TRON}) پیاده‌سازی شده اما هیچ یک بر روی سیستم‌های به همراه داده عمل نمی‌کنند. 

مورد قابل توجه دیگر افزوده شدن شرط خاتمه برای آزمون‌هاست (شرط $\xi$ در الگوریتم \ref{algorithm:enviocoFTCGen}). این شرط امکان این را فراهم می‌سازد تا در مورد اتمام مراحل آزمون تصمیم‌گیری شود. پیاده‌سازی این شرط نیز با معیارهای متفاوت و با توجه به نیاز صورت خواهد گرفت.

\item روشی برای مدل‌سازی داده‌های آزمون معرفی شده است که به طور کامل با الگوریتم تولید موارد آزمون هم‌خوان می‌باشد. این روش با استفاده از روش افراز رده‌ای و با اتکا به فرآورده‌های تولید نرم‌افزار و هم‌چنین دانش حاصل از متن برنامه، می‌تواند مقادیر داده‌ای را به همراه ارتباطات و محدودیت‌های بین آن‌ها مدل‌سازی نماید. استفاده از زبان مشترک یو‌ام‌ال برای توصیف این موارد تفاوت ماهوی این مدل‌ها را با مدل‌های رفتاری تا حد خوبی پوشش می‌دهد.

\item مفهوم سناریوهای آزمون، به هدف مشخص کردن مجموعه‌های آزمون که سیستم را در حضور محدودیت‌های محیطی مختلف مورد آزمون قرار می‌دهند، معرفی گردید. به علاوه، این سناریوها وظیفه‌ی مشخص نمودن مقادیر اولیه‌ی داده‌ای برای سیستم را به ازای هر مجموعه‌ی آزمون بر عهده گرفتند. این مسئله از این بابت حائز اهمیت است که الگوریتم ارائه شده در \cite{FTW05} در مورد تعیین مقادیر اولیه ($\iota$) برای ماشین حالت نمادین، چه در معناشناسی و چه در الگوریتم آزمون، اظهار نظر نمی‌کند اما ترکیب این مقادیر ممکن است بر رفتار سیستم اثری جدی داشته باشد. به این ترتیب استفاده از سناریو‌های آزمون می‌توانند راه‌کاری برای این مسئله باشد. در واقع سناریوهای آزمون را می‌توان به عنوان حلقه‌ی اتصال مدل‌های ساختار داده‌ای (که با روش‌هایی مانند افراز رده‌ای شکل می‌گیرند) و مدل‌های رفتاری (که از توصیف‌های رفتاری سیستم مثل موارد آزمون ساخته می‌شوند) دانست.

\item آزمون‌گری با پشتیبانی از ایده‌های ذکر شده طراحی و تولید شده است. این آزمون‌گر از توصیف‌های استاندارد \lr{XMI} (که یک استاندارد مبتنی بر \lr{XML} برای توصیف نمودارهای یوام‌ال است) بهره گرفته و موارد آزمون  مورد نظر را تولید می‌کند. در نهایت، نتیجه‌ی آزمون در قالب یک رأی اعلام می‌شود. با توجه به ضعف پشتیبانی ابزاری از روش آزمون مبتنی بر مدل به همراه داده، طراحی این آزمون‌گر از اهمیت ویژه‌ای برخوردار است.

\item کارایی روش پیشنهاد شده با مطالعه‌ی موردی یک سیستم واقعی، سنجیده شده است. اگرچه کارایی این روش را نمی‌توان تنها با استناد به یک مطالعه‌ی موردی نمایش داد،‌ با این حال استفاده از چهارچوب پیشنهادی و ابزارهای آن، تجربیات بسیار ارزنده‌ای در پی داشت که در برخی از مهم‌ترین ‌آن‌ها در فصل قبل بررسی شد. علاوه بر این در انتخاب مورد مطالعه تلاش شد تا نرم‌افزاری جامع در حوزه‌ی نرم‌افزارهای مالی (که خود گستره‌ی بزرگی را شامل می‌شود) انتخاب شود تا به این ترتیب بتوان به طور تقریبی از قابل استفاده بودن این روش در نمونه‌های دیگر نیز اطمینان حاصل نمود.
\end{strict_itemize}

\section{کاستی‌های چهارچوب}
چهارچوب پیشنهاد شده در این پژوهش دارای کاستی‌هایی نیز هست که کار بیشتری را می‌طلبد. در این بخش به طور فهرست‌وار به برخی از آن‌ها اشاره می‌کنیم:
\begin{itemize}
\item اگرچه تعریف \ref{definition:envdataioco} پایه‌ای نظری برای روال تولید موارد آزمون در حضور محدودیت‌های محیطی بنا می‌کند و در ادامه الگوریتم \ref{algorithm:final} روال تولید موارد آزمون را شرح می‌دهد، با این حال در این مرحله مقایسه‌ای بین قدرت بیان این تعریف از آی‌اوکو با تعریف اصلی آی‌اوکو به همراه داده (تعریف \ref{definition:dataioco}) انجام نمی‌شود. این مقایسه لازم است به طور نظری نشان دهد که آیا این دو تعریف، توصیف‌کننده‌ی یک رابطه‌ی مطابقت هستند و یا خیر و در صورت معادل نبودن در مورد جزئیات ارتباط این دو نیز بحث شود.

\item در این متن شرط خاتمه‌ی مجموعه‌ی آزمون ($\xi$) به طور کلی تعریف و در مورد برخی از معیارهای تعریف آن نیز بررسی اجمالی انجام شد. با این حال جزئیات تعریف هر یک از این معیارها، مقایسه‌ی آن‌ها از دید معناشناسی با یکدیگر و هم‌چنین روش پیاده‌سازی آن‌ها در این متن پوشانده نشده است. این صورت مسئله خود نیاز به پژوهش‌های مستقلی دارد چنان‌چه برخی از این معیارها هم‌اکنون در قالب پژوهش‌های دیگری در حال توسعه می‌باشند (برای مثال در \cite{Briones06} چهارچوبی برای اندازه‌گیری پوشش مدل ارائه شده است).

\item چهارچوب پیشنهاد شده در این متن تنها قادر است به صورت \gls*{در-لحظه} موارد آزمون را تولید کند. این روش اگرچه مزایای بسیاری دارد اما لزوماً در همه‌ی موارد بهترین گزینه نیست. برای مثال در حالت‌هایی که نیاز به \glspl*{آزمون بازگشتی}\LTRfootnote{regression tests} وجود دارد تولید هرباره‌ی حجم زیادی از موارد آزمون به صرفه نیست. در این موارد روش‌های تولید آزمون به صورت \gls*{برون‌خط} به همراه یک روش برای \gls*{اولویت‌بندی}\LTRfootnote{prioritization} موارد آزمون می‌تواند کارساز باشد. 

\item آزمون‌گر تولید شده، در حال حاضر از قابلیت \gls*{بررسی گونه‌ها}\LTRfootnote{type checking} برای گونه‌های داده‌ای طراحی شده پشتیبانی نمی‌کند. این امر سبب می‌شود تا امکان بروز خطای انسانی در طراحی مقادیر داده‌ای زیاد باشد. این مسئله با توجه به تنوع مقادیر مختلف داده‌ای که می‌تواند مورد استفاده قرار گیرد، بیشتر خودنمایی می‌کند.

\item به جز پشتیبانی از داده‌های آزمون، گسترش‌های دیگری از رابطه‌ی مطابقت آی‌اوکو وجود دارد. برای مثال آزمون سیستم‌های زمان‌دار توسط رابطه‌ی مطابقت $\mathbf{rtioco}$ توصیف شده و در ابزار \lr{Uppaal TRON} نیز پیاده‌ شده است \cite{Larsen04Tron}. برای تکمیل آزمون‌گر پیاده‌شده در این پژوهش لازم است پشتیبانی از این گسترش‌ها نیز به آن افزوده شود.

\item تبدیل انجام شده از زبان یوام‌ال به ماشین‌های گذار نمادین هنوز نادقیق است. دلیل این امر، گستردگی بسیار زیاد زبان یو‌ام‌ال و نیاز به تعریف نحو و معنای هر یک از اجزای آن در آزمون‌گر طراحی شده است. در این پژوهش، به طور ضمنی زیرمجموعه‌ی پشتیبانی شده از زبان یو‌ام‌ال معادل عناصری در نظر گرفته شده است که نحو و معنای آن‌ها در این متن تشریح شد.
\end{itemize}
 
\section{جهت‌گیری‌های پژوهشی آینده}
پژوهش حاضر، اگرچه ایده‌ای منسجم را در حوزه‌ی آزمون‌های مبتنی بر مدل دنبال می‌کرد، اما می‌توان آن را به عنوان نقطه‌ی آغازی برای چندین پژوهش مرتبط دانست. برخی ایده‌های مطرح شده در این متن هنوز ابتدایی هستند و برخی نیازها نیز با روش پیشنهادی قابلیت مدل‌سازی را ندارند. اگرچه تکمیل و رفع کاستی‌های ذکر شده در بالا را می‌توان به عنوان کارهای آینده در راستای این پژوهش دانست، اما علاوه بر آن‌ها برخی از جهت‌گیری‌های احتمالی برای پژوهش‌های آینده که از ایده‌ی این پژوهش بهره می‌برند را نیز می‌توان به شکل زیر برشمرد:
\begin{itemize}
\item ایده‌ی پیشنهادی در این پژوهش، راه را برای ایجاد یک \emph{فرآیند آزمون نرم‌افزار}\LTRfootnote{software testing process} کامل و جامع هموار می‌سازد. چنین فرآیندی لازم است در کنار فرآیند تولید نرم‌افزار قابل استفاده بوده و در آن ارتباط این دو نیز فرآیند نیز تعریف شود.

\item ایده‌ی افراز رده‌ای اگرچه چهارچوبی سطح بالا برای مدل‌سازی داده‌های آزمون و ارتباطات آن‌ها فراهم می‌کند. با این‌حال تولید مقادیر داده‌ای باید در این روش به صورت کاملاً دستی انجام شود. در ادامه لازم است روش‌های تولید رده‌ها و مقادیر داده‌ای به صورت نیمه‌خودکار و یا خودکار مورد نظر قرار گیرد.

\item بررسی مطالعه‌ی موردی انجام شده در این پژوهش، این ایده را تقویت می‌کند که ممکن است بتوان با استفاده از اطلاعات خاص دامنه‌ی سیستم (در مطالعه‌ی این پژوهش دامنه‌ی سیستم‌های مالی)، به روشی برای تولید توصیف‌های دقیق‌تر و موارد آزمون با کیفیت بالا دست یافت. بنابراین یک مسئله‌ی مهم که می‌تواند به عنوان جهت‌گیری احتمالی این پژوهش مورد نظر قرار گیرد، طراحی روش‌های آزمون و ابزارهایی است که به صورت \emph{\gls*{خاص-دامنه}}\LTRfootnote{domain specific} بهینه شده باشند.
\end{itemize}

