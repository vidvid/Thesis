به عنوان جمع بندی متن حاضر، در این فصل به فهرستی از مهم‌ترین دستاوردهای این پژوهش خواهیم پرداخت. در مورد هر یک از این دستاوردها برخی نکات مهم نیز ذکر شده است. بعد از این، برخی از مهم‌ترین کاستی‌های چهارچوب ارائه شده آورده شده است. این کاستی‌ها در هر دو جنبه‌ی نظری و عملی مورد بررسی قرار گرفته‌اند. در نهایت، بر مبنای این موارد، برخی جهت‌گیری‌های ممکن برای ادامه‌ی این پژوهش در آینده آورده شده است.

\section{دستاوردهای این پژوهش}
در این پژوهش روش طراحی منطق دامنه بر اساس تبادل ناهمگام پیغام مورد بررسی قرار گرفت. این روش طراحی با استفاده از مدل اکتور \cite{Agha_86} اشیاء سیستم را به فرایند‌های فعالی که قادر به تبادل پیغام با یکدیگر هستند تبدیل می‌کند. بررسی صورت گرفته در این پژوهش به هدف استخراج نکات و الگوهای طراحی و مقایسه‌ی آن با رویکرد طراحی شیءگرا به صورت ترتیبی انجام گرفته است. در  زیر برخی از مهمترین دستاوردهای این پژوهش آمده است:
\begin{itemize}
\item  یک سیستم نمونه انتخاب شده و طراحی منطق دامنه‌ی آن به روش تبادل ناهمگام پیغام به طور کامل انجام شده است. ارائه‌ی روش طراحی به صورت مرحله‌ای و افزایشی باعث شده است تا بتوان از آن به صورت دستورالعملی برای طراحی همروند استفاده کرد. 
\item خروجی مهم پژوهش، روش‌ها و الگوهایی است که در این نوع طراحی کاربرد دارد. در هر الگوی استخراج شده، روش پیاده‌سازی در مدل اکتور و کاربردهای الگو از نظر منطق دامنه بررسی شده است. 
\item تجربیاتی که در طراحی‌های صورت گرفته کسب شده به صورت قابل استفاده‌ای ارائه شده است و مطالعه‌ی این تجربیات، خواننده را با نکات ظریف و  حساسی آشنا می‌کند که انجام طراحی به روش تبادل ناهمگام پیغام را بسیار ساده‌تر می‌کند.
\item در ارزیابی روش طراحی ناهمگام، خصوصیات کیفی این روش از جمله تغییرپذیری و کارایی آن با روش طراحی شیءگرای ترتیبی مقایسه شده و نشان داده شده است که علاوه بر اینکه از نظر تغییرپذیری، دو روش قابل مقایسه هستند، طراحی به روش تبادل ناهمگام پیغام در مواردی باعث افزایش چشم‌گیر کارایی سیستم می‌گردد.
\end{itemize}
\section{جهت‌گیری‌های پژوهشی آینده}
برخی از جهت‌گیری‌های پژوهشی آینده برای تکمیل تحقیق حاضر در زیر آمده‌اند:
\begin{itemize}
\item در بررسی‌های صورت گرفته مشخص شد که برای ارزیابی کیفی طراحی شیءگرا به صورت ترتیبی معیارهای مختلفی وجود دارد که کیفیت برنامه را به صورت کمّی و قابل قیاس مشخص می‌کنند. با توجه به اینکه این معیارها بر اساس دیدگاه طراحی ترتیبی صورت گرفته و نکات و امکانات طراحی همروند در آنها نادیده گرفته شده است، نیاز به بازتعریف معیارهای موجود برای رویکرد طراحی بر اساس تبادل ناهمگام پیغام و نیز تعریف معیارهایی که مختص این رویکرد باشند کاملاً محسوس است. با توجه به نبود معیار‌های کیفیت مختص سیستم‌های شیءگرای همروند، در این پژوهش برای انجام مقایسه‌ی کیفی معیارهای مشابه‌ و قابل مقایسه با معیارهای طراحی ترتیبی استفاده شده است.
\item مورد دیگری که در پژوهش‌های آینده می‌تواند مورد توجه قرار بگیرد تدوین الگوهای طراحی در روش تبادل ناهمگام پیغام است. در طراحی شیءگرا به روش ترتیبی این الگوها به صورت مدوّن موجود هستند\cite{GOF}. پژوهش حاضر با ارائه‌ی تعدادی از الگوهای موجود قدمی در انجام این مهم برداشته است  اما مسلماً ارائه‌ی الگوهای طراحی در روش تبادل ناهمگام پیغام نیاز به بررسی پیاده‌سازی‌های متعدد در دامنه‌های مختلف دارد.  با توجه به تشابه ساختاری بین مدل اکتور و سیستم‌های عامل‌گرا\LTRfootnote{Agent Oriented}، بررسی جامع این دامنه جهت تطابق الگوهای طراحی ارائه شده در این حوزه می‌تواند منجر به تکمیل این پژوهش گردد. 
\item در این پژوهش، گام‌های برداشته شده در طراحی یک سیستم با رویکرد تبادل ناهمگام پیغام به صورت تجربی شرح داده شده است. تدوین مراحل دقیق این رویکرد در قالب یک متدولوژی طراحی می‌تواند به عنوان جهت‌گیری دیگری در پژوهش‌های آینده مورد بررسی قرار گیرد. با توجه به تشابه‌‌های سیستم‌های مبتنی بر اکتور با سیستم‌های عامل‌گرا، در این حوزه می‌توان از متدولوژی‌های عامل‌گرا به عنوان الگو و مرجع استفاده کرد.
\end{itemize}

