به عنوان جمع بندی متن حاضر، در این فصل به فهرستی از مهم‌ترین دستاوردهای این پژوهش خواهیم پرداخت. در مورد هر یک از این دستاوردها برخی نکات مهم نیز ذکر شده است. بعد از این، برخی از مهم‌ترین کاستی‌های چهارچوب ارائه شده آورده شده است. این کاستی‌ها در هر دو جنبه‌ی نظری و عملی مورد بررسی قرار گرفته‌اند. در نهایت، بر مبنای این موارد برخی جهت‌گیری‌های ممکن برای ادامه‌ی این پژوهش در آینده آورده شده است.

\section{دستاوردهای این پژوهش}
این پژوهش، چهارچوبی بدیع برای آزمون سیستم‌های نرم‌افزاری بر پسیستم واقعی استفاده می‌شود.

در واقع چهارچوب پیشنهاد شده تلاش می‌کند تا مجموعه‌ی به هم‌پیوسته‌ای از فعالیت‌ها برای آزمون را، از اولین مراحل طراحی تا نتیجه‌گیری از مجموعه‌ی آزمون‌ها، پیشنهاد کند. در زیر برخی از مهم‌ترین دستاورد‌های هر یک از مراحل این کار آمده است:

\section{کاستی‌های چهارچوب}
چهارچوب پیشنهاد شده در این پژوهش دارای کاستی‌هایی نیز هست که کار بیشتری را می‌طلبد. در این بخش به طور فهرست‌وار به برخی از آن‌ها اشاره می‌کنیم:
\section{جهت‌گیری‌های پژوهشی آینده}
بهره می‌برند را نیز می‌توان به شکل زیر برشمرد:

