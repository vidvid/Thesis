در فصل قبل اجزای چهارچوب پیشنهادی این پژوهش به تفصیل تشریح شد و در م
\section{روش ارزیابی}
\section{ارزیابی کارایی}
سبیس
\section{ارزیابی تغییرپذیری}
سبیس
\subsection{بررسی معیارهای  ایستا }
با توجه به بزرگی سیستم مورد مطالعه، برای این مطالعه‌ی موردی دو مورد کاربرد از مجموعه‌ی مهم‌تری
\subsection{اعمال تغییرات}
\subsubsection{تغییر اول}

\section{نتایج ارزیابی}
قبل از بررسی نتایج، لازم است برخی نکات در مورد اجرای آزمون‌ها مورد بررسی قرار گیرد. مطابق آن‌چه در فص

\subsection{تحلیل نتایج}
با داشتن نتای