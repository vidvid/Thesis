در فصل قبل اجزای چهارچوب پیشنهادی این پژوهش به تفصیل تشریح شد و در م
\section{روش ارزیابی}
\section{ارزیابی کارایی}
سبیس
\section{ارزیابی تغییرپذیری}
سبیس
\subsection{بررسی معیارهای  ایستا }
با توجه به بزرگی سیستم مورد مطالعه، برای این مطالعه‌ی موردی دو مورد کاربرد از مجموعه‌ی مهم‌تری
\subsection{اعمال تغییرات}
\subsubsection{تغییر اول}

\section{نتایج ارزیابی}
قبل از بررسی نتایج، لازم است برخی نکات در مورد اجرای آزمون‌ها مورد بررسی قرار گیرد. مطابق آن‌چه در فصل قبل بیان شد، برای اجرای متوالی مجموعه‌های آزمون مختلف لازم است معیاری برای خاتمه‌ی هر مجموعه‌ی آزمون معرفی شود. با توجه به یکسان بودن احتمال بروز همه‌ی سناریوهای رفتاری در این مطالعه‌ی موردی و هم‌چنین تجربیات حاصل از چند اجرای آزمون‌ مقدماتی، در این مورد خاص از معیار زمانی برای خاتمه‌ی هر مجموعه و اجرای مجموعه‌ی بعدی استفاده شده است. اگر چه این معیار، ابتدایی‌ترین شرط خاتمه محسوب می‌شود، اما درعوض ساده‌ترین و کم هزینه‌ترین معیار نیز محسوب می‌شود و همان‌طور که اشاره شد در مورد جاری نتایج نسبتاً قابل قبولی نیز تولید می‌کند.

برای ارزیابی آزمون‌ها و مقایسه‌ی حالت‌های مختلف در این مطالعه‌ی موردی، از معیار پوشش متن\LTRfootnote{code coverage}  استفاده شده است. به این ترتیب مجموعه‌ی آزمونی بهتر فرض می‌شود که پوشش بالاتری از متن برنامه دارد. برای اندازه‌گیری این معیار نیز از ابزار متن باز \lr{Cobertura}\cite{cobertura} استفاده شده است. 

برای ارائه‌ی تحلیلی از کارایی روش معرفی شده در این پژوهش، نتایج آزمون‌های تولید شده توسط مدل‌های بخش قبل (پوشش متن حاصل از انجام آزمون‌های تولید شده بر مبنای این روش) باید با روش شناخته شده‌ی دیگری مقایسه شود. در این مطالعه‌ی موردی این نتایج با نتایج استفاده از روش اولیه‌ی آزمون مبتنی بر مدل مقایسه شده است. یادآوری این نکته ضروری است که الگوریتم آزمون مبتنی بر مدل استاندارد، امکان تبادل داده با سیستم را در طول آزمون ندارد. از آن‌جا که سیستم مورد این مطالعه به طور ذاتی مبتنی بر داده است، غنی کردن آزمون‌های رفتاری تولید شده از روش آزمون مبتنی بر مدل استاندارد، در آداپتور پیاده‌شده برای آن انجام شده است. به این معنی که مقادیر داده‌ای مربوطه پیش از اجرا در آداپتور نوشته شده و به ازای هر پیغام دریافت شده از آزمون‌گر، آداپتور با افزودن این مقادیر به پیغام، آن را کامل کرده و برای سیستم ارسال می‌کند. در مورد پاسخ‌های دریافتی از سیستم نیز مشابه همین روال برای پیغام‌های دریافت شده اتفاق می‌افتد. واضح است که با استفاده از این ایده فقط یک مقداردهی داده‌ای مورد آزمون قرار می‌گیرد.

با استفاده از این روش، آزمون‌های تولید شده بر پایه‌ی روش پیشنهادی این پژوهش با نتایج دو آزمون دیگر مقایسه می‌شوند: یکی آزمونی مبتنی بر آی‌اوکو استاندارد که داده‌های تعبیه شده در آن منجر به انجام کامل عملیات در سیستم (اجرای سناریوی اصلی مورد کاربرد) می‌شود؛ و دیگری آزمونی مبتنی بر آی‌اوکو استاندارد که داده‌های تعبیه شده در آن منجر به بروز خطا و اتمام ناقص عملیات در سیستم (اجرای یکی از سناریوهای فرعی مورد کاربرد) می‌شود. این کار برای هر سه مجموعه‌ی آزمون ذکر شده در بخش قبل انجام شده و نتایج در جدول \ref{table:testResults} ذکر شده است.


\subsection{تحلیل نتایج}
با داشتن نتای