در این فصل به طور اجمالی مروری بر پیش‌زمینه‌ی پژوهش انجام شده است. در هر بخش سعی شده است که با حفظ اختصار، تنها جنبه‌های  کاربردی مرتبط با پژوهش مطرح گردد.
\section{مدل بازیگر}

\gls{مدل بازیگر}%
، که توسط هیوئیت و آقا \cite{Hewitt1972,Agha1987,Agha1990} ایجاد شده‌است، یک نمایش سطح بالا از سیستم‌های توزیع‌شده فراهم می‌کند. 
\gls{بازیگر}ها
اشیای \gls{لفافه‌بندی‌شده}‌ای هستند که به صورت \gls{همروند} فعالیت می‌کنند و دارای \gls{رفتار}\LTRfootnote{Behavior} قابل تغییر هستند. 
بازیگرها \gls{حالت  مشترک}\LTRfootnote{Shared State} ندارند و تنها راه ارتباط بین آنها تبادل ناهمگام پیغام است. 
 در مدل اکتور فرضی در مورد مسیر پیغام و میزان تاخیر در رسیدن پیغام وجود ندارد، در نتیجه ترتیب رسیدن پیغام‌ها \gls{غیرقطعی} است.
 در یک دیدگاه می‌توان بازیگر را یک \gls{شی‌ء} در نظر گرفت که به یک ریسمان\gls{ریسمان}\LTRfootnote{ُThread} کنترل، یک صندوق پست و یک نام غیر قابل تغییر و به صورت سرارسی یکتا \LTRfootnote{Globally Unique} در نظر گرفت. برای ارسال پیغام به یک بازیگر، از نام آن استفاده می‌شود. در این مدل، نام  یک بازیگر را می‌توان در قالب پیغام  ارسال کرد.
 
 یک بازیگر در نتیجه‌ی دریافت پیغام احتمالا محاسباتی انجام می‌دهد و در نتیجه‌ی آن یک از ۳ عمل زیر را انجام می‌دهد:
\begin{itemize}
\item[ارسال پیغام]
\item[ایجاد بازیگر جدید]
\item[تغییر حالت محلی]

\end{itemize} 
از نظر \gls{معناشناسی}LTRfootnote{Semantics} مشخصات کلیدی مدل اکتور عبارتند از: لفافه‌بندی حالت،  زمانبندی منصفانه‌ی بازیگرها و تبادلات پیغام‌ها، و نامگذاری جهانی\LTRfootnote{universal naming}
 
پاسخگویی به هر پیام شامل برداشتن آن پیام از صندوق پستی و اجرای عملیات متناسب با آن است.
این اجرای عملیات به صورت \gls{تجزیه‌ناپذیر}\LTRfootnote{Atomic} و بی‌وقفه خواهد بود.

همان گونه که گفته‌شد، مدل بازیگر سیستم را در سطح بالایی از انتزاع مدل می‌کند.
این ویژگی دامنهٔ سیستم‌های قابل مدلسازی توسط مدل بازیگر را بسیار وسیع نموده‌است.
انواع سیستم‌های سخت‌افزاری و نرم‌افزاری طراحی‌شده برای زیرساخت‌های خاص یا عام، و همچنین الگوریتم‌ها و پروتکل‌های توزیع‌شدهٔ مورد استفاده در شبکه‌های ارتباطی از جملهٔ موارد مناسب برای بهره‌گیری از مدل بازیگر هستند.

تأکید ویژهٔ مدل بازیگر بر ارسال ناهمگام پیام نیز یکی از ویژگی‌های این مدل است که آن را برای مدلسازی سیستم‌های توزیع‌شده مناسب کرده‌است.
این مکانیزم ارسال پیام می‌تواند برای مدلسازی ارتباطات شبکه‌ای و تیادل اطلاعات در محیط‌های توزیع‌شده مورد استفاده قرار گیرد.
در حقیقت مدل بازیگر با توجه به اینکه انتزاعی سطح بالا از همهٔ مؤلفه‌های موجود در سیستم‌های توزیع‌شده است، بسیار نزدیک به واقعیت است و در مواردی که سیستم مورد نر یک سیستم توزیع‌شده با مؤلفه‌های موازی باشد کار مدلسازی را بسیار تسهیل می‌کند.




\subsection{جایگاه آزمون مبتنی بر مدل در میان روش‌های بازرسی کارکرد نرم‌افزار}
روش‌های زیادی برای بازرسی سیستم‌های نرم‌افزاری به هدف اطمینان از کارکرد صحیح آن‌ها ارائه شده است. اگرچه نمی‌توان به طور مشخص در مورد برتری یکی از این روش‌ها به دیگری نظر داد اما می‌‌توان آن‌ها را از نظر شیوه‌ی استفاده طبقه‌بندی نمود. در زیر، یک طبقه‌بندی برای روش‌های آزمون مبتنی بر مدل، با توجه به نوع محصولات مورد استفاده و روش استفاده از آن‌ها، آمده است.

\begin{description}
\item[روش‌های مبتنی بر بررسی مدل:]
بالای بازرسی مدل‌ها به دلیل نیاز به ساخت فضاهای حالت به طور کامل و پیمایش آن‌هاست.
\item[روش‌های ]\LTRfootnote{static analysis} در این روش‌ها خصوصیات مورد نظاده از این روش‌ها را به عنوان جایگزین 
\end{description}

آزمون مبتنی بر مدل با بهره‌گیری از ایده‌ی مدل‌سازی از روش‌های تحلیل مدل از یک سو و اجرای موارد آزمون تولید شده از سوی دیگر تلاش می‌کند گونه‌ای کارا از آزمون نرم‌افزار را ارائه نماید. آزمون مبتنی بر مدل با ادر ادامه‌ی متن حاضر برای راحتی سیستم نامیده خواهد شد) اجرا می‌شوند. 

\subsection{؟؟}
\label{subsection:ioco}
رابطه مطابقت موجود و اعلام نظر در مورد مطابقت و یا عدم مطابقت این دو ارائه می‌کند.


\section{طراحی مبتنی بر دامنه}
\label{section:CategoryPartitionMethod}
اگرچه رابطه‌ی معرفی شده از مطابقت ماشین‌های گذار نمادین امکان بازرسی رفتار سیستم به همراه داده را فراهم می‌کند، با این‌حال این روش مشخص نمی‌کند که مقادیر داده‌ای که باید مورد آزمون قرار بگیرند از چه طریقی باید مشخص شوند. 
 
بعد از طی این مراحل و تکمیل اطلاعات در مورد مقادیر داده‌ای، در نهایت موارد آزمون نهایی با جایگزینی گزینه‌های انتخاب شده از رده‌های مختلف و اجرای رفتار مشخص شده‌ی آزمون برای هر واحد کارکردی، تولید می‌شود.

\subsubsection{؟}
اگرچه روش افراز داده‌ای راه‌حلی سطح بالا برای رده‌بندی داده‌های آزمون به دست می‌دهد، در این روش تمامی گزینه‌های انتخاب شده باید در یک سطح بیان شوند و امکان طبقه‌بندی داده‌ای در آن وجود ندارد (به این معنی که نمی‌توان یک رده‌ی انتخاب شد
