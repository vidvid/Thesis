در این فصل به ارائه‌ی برخی کارهای پیشین و مرتبط به موضوع این پژوهش خواهیم پرداخت. در مورد هر یک از این موارد به ارتباط آن با بحث جاری، کاربرد و یا نقاط تأثیرگذار آن در موضوع این پژوهش و هم‌چنین ضعف ها و نقایص آن‌ها پرداخته شده است. 

\section{الگوهای همگام سازی}
\label{section:coordinationPatterns}
ابزارهای گوناگونی برای آزمون خودکار نرم‌افزار تولید و عرضه شده‌اند که بر پایه‌های مختلف نظری مبتنی هستند. تکیه‌ی ما در این بحث بر روش‌هایی است که از آی‌اوکو به عنوان روش رابطه‌ی مطابقت استفاده می‌کنند. 
\section{طراحی به روش ارتباط ناهمگام}
\label{section:asyncDesign}
این بخش به بررسی روش‌ها و ابزارهایی می‌پردازد که داده‌‌ها را به عنوان ورودی و خروجی سیستم در نظر می‌گیرند و با تولید ترکیب‌های مختلف از مقایر داده‌ای و رد و بدل کردن آن با سیستم به انجام وارسی می‌پردازند.

\subsection{مالتی کور}
لف از سه نمادگذاری مختلف پشتیبانی می‌کند. \textit{نمودارحالت یوام‌ال، نمادگذاری بی\LTRfootnote{B Notation} و نمودار کلاس یوام‌ال}. در این ابزار نمودار حالت بیشتر برای توصیف رفتارهای سیستم مورد استفاده قرار می‌گیرد در حالی که توصیف نمادگذاری بی بیشتر برای نمایش ساختارهای داده‌‌ای مورد استفاده قرار می‌گیرد. 
خلاصه نمود:
\begin{enumerate}
\item تولید حالت‌های مختلف رفتار سیستم با استفاده از مدل رفتاری داده‌شده به سیستم (مانند ماشین‌های حالت)
\item محاسبه‌ی مقادیر داده‌ای به هدف یافتن حالت‌های مرزی به ازای هر یک از رفتارهای سیستم
\item تولید موارد آزمون با استفاده از حالت‌های مرزی و معرفی ترتیب‌هایی از عملیات که در شرط‌های مرزی صدق می‌کنند
\end{enumerate}
شکل \ref{fig:LTGSample}.(الف) نمونه‌ای از توصیف یک ساختار داده‌ای مورد آزمون در نمادگذاری بی را نشان می‌خی از این موارد می‌پردازیم:
\begin{itemize}
\item آزمون‌گر ال‌تی‌جی اگرچه از تولید خودکار موارد آزمون از روی مدل رفتاری سیستم پشتیبانی می‌کند، اما الگوریتم تولید موارد آزمون در آن مبتنی بر رابطه‌های مطابقت (مانند آی‌اوکو) نیست. نتیجه‌ی عدم پشتیبانی از آی‌اوکو ایجاد محدودیت‌هایی بر روی روش مدل‌سازی است. برای مثال، مدل‌های ورودی ال‌تی‌جی باید همگی متناهی و قطعی باشند در حالی که در آزمون‌گرهایی مانند تورکس چنین محدودیت‌هایی وجود ندارد.
\item علی‌رغم پشتیبانی از مقادیر داده‌ای، در ال‌تی‌جی امکان تعریف و توصیف گونه‌های داده‌ای وجود ندارد. این امر باعث می‌شود که محدودیت‌های زیادی در توصیف انواع گوناگون داده به وجود آید.
\item برای مدل‌سازی جنبه‌های مختلف سیستم در ال‌تی‌جی نیاز به استفاده از نمادگذاری‌های مختلفی است که این امر هم باعث دشواری در نگارش توصیف‌ها و خوانایی و هم دشواری در نگه‌داری مدل‌ها می‌شود.
\item این آزمون‌گر امکان تولید موارد آزمون به صورت در-لحظه را ندارد و موارد آزمون تولید شده باید ابتدا تولید و نگه‌داری شده و سپس برای آزمون سیستم مورد استفاده قرار گیرند.
\end{itemize}

سلام