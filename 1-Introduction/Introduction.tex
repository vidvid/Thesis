از دیدگاه مهندسی نیازمند‌ی ن
\section{انگیزه‌ی پژوهش}
روش‌های آزمون مبتنی بر مدل کاستی‌هایی نیز دارند. برای مثال در آی‌اوکو، استفاده از ماشین گذار برای توصیف سیستم تحت آزمون می‌تواند منجر به تولید مدل‌های بسیار پیچیده‌ای شود، زیرا ماشین‌های گذار علی‌رغم قدرت بیان بالا، چنا‌ن‌چه خواهیم دید، از نظر توصیفی نمادگذاری سطح پایینی محسوب می‌شوند و بنابراین مدل‌سازی جزئیات سیستم ممکن است حجم زیادی از پیچیدگی را در مدل‌ها به وجود آورد.

\section{صورت مسئله}
تمرکز اصلی این پژوهش بر آزمون مبتنی بر مدل سیستم‌های وابسته به داده\LTRfootnote{data dependent} قرار دارد. سیستم‌های وابسته به داده معمولاً حجم زیادی از اطلاعات را با محیط خود مبادله‌ می‌‌کنند و رفتار آن‌ها وابسته به محاسباتی است که بر روی مقادیر داده‌ای انجام می‌دهند. 
 
\section{روش پژوهش}
ارزیابی عملی با مطالعه موردی
\section{روش ارزیابی}
GQM
\section{خلاصه‌ی دستاوردهای پژوهش}
برخی از دستاوردهای این پژوهش را می‌توان به این ترتیب برشمرد:
داد. به این منظور نمای سطح بالا برای تولید آزمون‌ها و بررسی نتایج در روش اصلی آزمون مبتنی برمدل در شکل

\section{ساختار پایان‌نامه}
برای بررسی این موارد، ساختار این متن در ۶ فصل تنظیم گردیده است:
\begin{strict_itemize}
\item
 به طور خلاصه مورد بررسی قرار گرفته‌اند.
\item
 بررسی 
\item
ه‌اند.
\item
\end{strict_itemize}
