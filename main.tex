\documentclass[oneside, a4paper,11pt]{book}
\usepackage{graphicx}
\usepackage{wrapfig}
\usepackage{verbatim}
\usepackage{fancyhdr}
\usepackage{url}
\usepackage{supertabular}
\usepackage{multicol}
\usepackage{setspace}

\usepackage{array}
\usepackage{colortbl}

\usepackage{amsmath}
\usepackage{amsthm}
\usepackage{amssymb}
\usepackage{cancel}
\usepackage{mathtools}
\usepackage{extpfeil}
%\usepackage[only, llbracket, rrbracket]{stmaryrd}


\usepackage[pagebackref=false]{hyperref}
\usepackage[rgb,x11names,dvipsnames]{xcolor}% Optimize for screen reading.
\usepackage{longtable}
\usepackage{listings}
\lstdefinelanguage{scala}{
  morekeywords={import, object, class, case, var, val, if, else, while, for, from, to, extends, def, self, false, true, now, loop, react, ruturn, override},
  otherkeywords={=>,<-,::},
  sensitive=true,
  morecomment=[l]{//},
  morecomment=[n]{/*}{*/},
  morestring=[b]",
  morestring=[b]',
  morestring=[b]"""
}

\lstset{frame=tb,
  language=scala,
  aboveskip=0mm,
  belowskip=0mm,
  showstringspaces=false,
  columns=flexible,
  basicstyle={\small\ttfamily},
  keywordstyle=\color{RedViolet},
  numbers=left,
  numberstyle=\tiny\color{blue},
  numbersep=5pt,
  stringstyle=\color{Dandelion},
  stepnumber=1,
  frame=none,
  breaklines=true,
  breakatwhitespace=true,
  tabsize=2
  float=[hptb]
}
\usepackage[xindy]{glossaries}
\usepackage[margin=10mm,font={footnotesize},labelfont={footnotesize}]{caption}
\usepackage[section]{placeins}
\usepackage{xepersian}

\hypersetup{
   colorlinks,%
   filecolor=black,%
   linkcolor=DeepSkyBlue4,
  urlcolor=Chocolate4,
  citecolor=Cyan4,
    bookmarks=true,         
  %  unicode=true,          
    pdfstartview={FitH},    
    pdftitle={Design of Domain Using Asynchronous Message Passing},    
    pdfsubject={طراحی منطق دامنه بر اساس تبادل ناهمگام پیغام}
    pdfauthor={وحید ذوقی},  
    pdfsubject={پایان‌نامه‌ی کارشناسی ارشد},  
    pdfcreator={Vahid Zoghi},   
    pdfkeywords={Asynchronous Message Passing, Domain Modeling, Concurrency, Design Patterns}, 
}

\numberwithin{equation}{chapter}
\numberwithin{table}{chapter}
\numberwithin{figure}{chapter}
\numberwithin{equation}{chapter}

\settextfont[Scale=1.2]{XB Niloofar}
\setlatintextfont[Scale=1.2]{Times New Roman}
\DeclareMathSizes{10}{12}{8}{6}   % For size 10 text
\defpersianfont\shafigh[Scale=1]{Times New Roman}


\linespread{2}
\setlength\parskip{0.25cm}
\setlength\topmargin{-0.5in}
\setlength\headheight{2cm}
\setlength\headsep{0.7cm}
\setlength\textheight{8.8in}
\setlength\textwidth{6.5in}
\setlength\oddsidemargin{-0.3in}
\setlength\evensidemargin{0.0in}


\newenvironment{strict_enumerate}
{\begin{enumerate}
  \setlength{\itemsep}{1pt}
  \setlength{\parskip}{0pt}
  \setlength{\parsep}{0pt}}
{\end{enumerate}}

\newenvironment{strict_itemize}
{\begin{itemize}
  \setlength{\itemsep}{6pt}
  \setlength{\parskip}{0pt}
  \setlength{\parsep}{0pt}}
{\end{itemize}}


\newenvironment{strict_description}
{\begin{description}
  \setlength{\itemsep}{6pt}
  \setlength{\parskip}{0pt}
  \setlength{\parsep}{0pt}}
{\end{description}}

\newtheorem{definition}{تعریف}[chapter]
\newtheorem*{definition*}{تعریف}
\newtheorem{algorithm}{الگوریتم}[chapter]
\newtheorem{theorem}{قضیه}[chapter]
\newtheorem{lemma}{لم}[chapter]

%\let\lstinputlistingO=\lstinputlisting
\newcommand{\codelisting}[4][]
{
\begin{figure*}
\begin{latin}
\begin{center}
\begin{tabular}[hbtp]{c}

\linespread{1.4}
\lstinputlisting[#1]{#2}

\end{tabular}
\end{center}
\end{latin}
\caption{#3}
\label{#4}
\end{figure*}
}
%\newcommand{\codelstinputlisting}[2]{
%\begin{latin}
%linespread{6}
%\lstinputlisting[{#1}]{{#2}}
%\end{latin}
%}




\newglossarystyle{mylist}{%
% put the glossary in the itemize environment:
\renewenvironment{theglossary}{}{}%
% have nothing after \begin{theglossary}:
\renewcommand*{\glossaryheader}{}%
% have nothing between glossary groups:
\renewcommand*{\glsgroupheading}[1]{}%
\renewcommand*{\glsgroupskip}{}%
% set how each entry should appear:
\renewcommand*{\glossaryentryfield}[5]{%
%\item[] % bullet point
\glstarget{##1}{##2}% the entry name
\dotfill% the symbol in brackets
\space ##3 \\% the number list in square brackets
}%
% set how sub-entries appear:
\renewcommand*{\glossarysubentryfield}[6]{%
\glossaryentryfield{##2}{##3}{##4}{##5}{##6}}%
}

\makeglossaries
\newglossaryentry{بازیگر}{name={بازیگر}, description={\lr{actor}}}
\newglossaryentry{مدل بازیگر}{name={مدل بازیگر}, description={\lr{actor model}}}
\newglossaryentry{لفافه‌بندی‌شده}{name={لفافه‌بندی‌شده}, description={\lr{encapsulated}}}
\newglossaryentry{رفتار}{name={رفتار}, description={\lr{behavior}}}
\newglossaryentry{تجزیه‌ناپذیر}{name={تجزیه‌ناپذیر}, description={\lr{atomic}}}
\newglossaryentry{انصاف}{name={انصاف}, description={\lr{fairness}}}
\newglossaryentry{حالت}{name={حالت}, description={\lr{state}}}
\newglossaryentry{حالت مشترک}{name={حالت مشترک}, description={\lr{shared state}}}
\newglossaryentry{تجزیه}{name={تجزیه}, description={\lr{decomposition}}}
\newglossaryentry{شیءگونه}{name={شیءگونه}, description={\lr{object-style}}}
\newglossaryentry{شیء-بنیاد}{name={شیء-بنیان}, description={\lr{Object-Based}}}
\newglossaryentry{استدلال}{name={استدلال}, description={\lr{Reason}}}
\newglossaryentry{بی‌قاعده}{name={بی‌قاعده}, description={\lr{Irregular}}}
\newglossaryentry{مقیاس‌پذیر}{name={مقیاس‌پذیر}, description={\lr{Scalable}}}
\newglossaryentry{پراکنده}{name={پراکنده}, description={\lr{Sparse}}}
\newglossaryentry{حالت محلی}{name={حالت محلی}, description={\lr{Local State}}}
\newglossaryentry{واکنشی}{name={واکنشی}, description={\lr{Reactive}}}
\newglossaryentry{همگام}{name={همگام}, description={\lr{Synchronous}}}
\newglossaryentry{ناهمگام}{name={ناهمگام}, description={\lr{Asynchronous}}}
\newglossaryentry{شی‌ء}{name={شی‌ء}, description={\lr{Object}}}
\newglossaryentry{ریسمان}{name={ریسمان}, description={\lr{Thread}}}
\newglossaryentry{میانگیری‌شده}{name={میانگیری‌شده}, description={\lr{Buffered}}}
\newglossaryentry{زمان‌بندی}{name={زمان‌بندی}, description={\lr{Scheduling}}}
\newglossaryentry{ارلانگ}{name={ارلانگ}, description={\lr{Erlang}}}
\newglossaryentry{بررسی گونه‌ها}{name={بررسی گونه‌ها}, description={\lr{type checking}}}
\newglossaryentry{مشخصه}{name={مشخصه}, plural={مشخصات},description={\lr{property}}}
\newglossaryentry{پیاده‌سازی}{name={پیاده‌سازی}, description={\lr{implementation}}}
\newglossaryentry{گونه}{name={گونه}, description={\lr{type}}}
\newglossaryentry{میان‌گذاری}{name={میان‌گذاری}, description={\lr{interleaving}}}
\newglossaryentry{همروند}{name={همروند}, description={\lr{Concurrent}}}
\newglossaryentry{غیرقطعی}{name={غیرقطعی}, description={\lr{Non-deterministic, Indeterminate}}}
\newglossaryentry{رده}{name={رده}, description={\lr{category}}}
\newglossaryentry{گزینه}{name={گزینه}, description={\lr{choice}}}
\newglossaryentry{درخت‌ طبقه‌بندی}{name={درخت‌ طبقه‌بندی}, description={\lr{classification tree}}}
\newglossaryentry{مورد کاربرد}{name={مورد کاربرد}, description={\lr{use case}}, plural={موارد کاربرد}}
\newglossaryentry{بسته‌}{name={بسته‌}, description={\lr{package}}}
\newglossaryentry{رویداد}{name={رویداد}, description={\lr{event}}}
\newglossaryentry{حوزه‌}{name={حوزه‌}, description={\lr{scope}}}
\newglossaryentry{بازه}{name={بازه}, description={\lr{range}}}
\newglossaryentry{نمونه}{name={نمونه}, description={\lr{instance}}}
\newglossaryentry{کنش}{name={کنش}, description={\lr{action}}}
\newglossaryentry{ناحیه}{name={ناحیه}, description={\lr{region}}}
\newglossaryentry{بازمتن}{name={متن‌باز}, description={\lr{open source}}}
\newglossaryentry{مطابقت}{name={مطابقت}, description={\lr{conformance}}}
\newglossaryentry{تراکنش‌}{name={تراکنش‌}, description={\lr{transaction}}}
\newglossaryentry{منطق دامنه}{name={منطق دامنه}, description={\lr{domain logic}}}
\newglossaryentry{آزمون واحد}{name={آزمون واحد}, description={\lr{unit test}}}
\newglossaryentry{خاص-دامنه}{name={خاص-دامنه}, description={\lr{domain specific}}}
\newglossaryentry{یک‌پارچه}{name={یک‌پارچه}, description={\lr{integrated}}}
\newglossaryentry{روش صوری}{name={روش صوری}, description={\lr{formal method}}, plural={روش‌های صوری}}
\newglossaryentry{جفت‌شدگی}{name={جفت‌شدگی}, description={\lr{coupling}}}
\newglossaryentry{گسترش}{name={گسترش}, description={\lr{extension}}}
\newglossaryentry{معناشناسی}{name={معناشناسی}, description={\lr{Semantics}}}
\newglossaryentry{تابعی}{name={تابعی}, description={\lr{functional}}}
\newglossaryentry{ایستا}{name={ایستا}, description={\lr{static}}}





\begin{document}

% Besmellah Page
\newpage
\thispagestyle{empty}
\begin{tabular}{c}
\vspace{5cm}\\
\includegraphics[width=16cm]{Figures/besm.pdf}\\
\end{tabular}
%


\newpage
\thispagestyle{empty}
\mbox{}



%%%%%%%%%%%%%%%%%
%TITLE PAGE
%%%%%%%%%%%%%%%%%
\newpage
\thispagestyle{empty}
\begin{center}
\begin{tabular}{lp{7cm}r}
\includegraphics[width=2.8cm]{Figures/utlogo} & & \includegraphics[width=3.8cm]{Figures/englogo} \\
\end{tabular}

{\LARGE\bfseries دانشگاه \space تهران}
\\*
{\Large\bfseries پردیس \space دانشکده‌های فنی}
\\*
{\Large\bfseries دانشکدهٔ \space مهندسی برق و کامپیوتر}
\par
\vskip 1.5cm
{\Huge\bfseries طراحی منطق دامنه بر اساس تبادل ناهمگام پیغام}\par
\vskip 1cm
{\large%
  نگارش }\par
{\Large\bfseries وحید ذوقی شال}\par
\par
{\large
  استاد راهنما\par
\Large\bfseries دکتر رامتین خسروی}
\par
\vskip 1.5cm
{\large\bfseries پایان‌نامه برای دریافت درجهٔ \space کارشناسی ارشد \space در رشتهٔ \\* مهندسی کامپیوتر - گرایش نرم‌افزار}
\par
\vskip 1cm
{\large شهریور ۱۳۹۱}
\par
\vfill
\end{center}
%%%%%%%%%%%%%%%%%%%%


\newpage
\thispagestyle{empty}
\mbox{}

\newpage
\thispagestyle{empty}
\begin{flushleft}   
\begin{tabular}{l} 
\vspace{5cm} \\
\shafigh{ تقدیم به پدرم، مادرم و همسر مهربانم} \\
\end{tabular}
\end{flushleft}

\newpage
\thispagestyle{empty}
\mbox{}

\newpage
\thispagestyle{empty}
\begin{center}
\large{\bfseries{قدردانی}}\\
\end{center}
‎در ابتدا لازم می‌دانم از جناب آقای دکتر رامتین خسروی که در انجام این پژوهش  افتخار استفاده از راهنمایی ایشان را داشتم، تشکر و قدردانی کنم. مطمئناً این کار بدون کمک‌های همه‌جانبه و بی‌شائبه‌ی ایشان امکان‌پذیر نبود. از اعضای هیئت داوران محترم نیز برای فرصتی که در اختیار من قرار دادند تشکر می‌کنم.

\newpage
\thispagestyle{empty}
\mbox{}


\pagestyle{plain}

\newpage
{\centering\large{\bf{طراحی منطق دامنه بر اساس تبادل ناهمگام پیغام}} \par}
\subsection*{چکیده}
در سال‌های اخیر گرایش به مدل اکتور چه در دنیای پژوهش و چه در صنعت افزایش پیدا کرده است. تغییر روند افزایش سرعت پردازنده‌ها به سمت افزایش تعداد هسته‌ها، استفاده از زیرساخت‌های محاسبات ابری و گرایش به تولید برنامه‌های توزیع شده می‌توانند از جمله‌ی دلایل این علاقه‌مندی باشند. از سوی دیگر علیرغم وجود منابع گسترده برای یادگیری طراحی به روش شیءگرا، کمبود پژوهش در زمینه‌ی روش‌ها و نکات موجود در طراحی شیءگرای همروند محسوس می‌باشد. در این پژوهش تلاش شده است تا با انجام طراحی یک سیستم انتخاب شده با استفاده از تبادل ناهمگام پیغام، روش‌ها،‌ الگوها و نکات موجود در این روش طراحی بررسی شده و به صورت قابل استفاده‌ای ارائه گردند. طراحی انجام شده با استفاده از معیارهای کیفی نرم‌افزار، با طراحی شیءگرای عادی (ترتیبی) مقایسه شده و نشان داده شده است که از نظر کیفی این طراحی قابل مقایسه و در مواردی بهتر از طراحی ترتیبی است. علاوه بر این با استفاده از این نوع طراحی،‌ همروندی ذاتی در سیستم ایجاد می‌شود و قابلیت توزیع برنامه به دلیل خصوصیات معنایی مدل اکتور به صورت قابل توجهی افزایش می‌یابد.\\

\textbf{واژه‌های کلیدی: }\textit{ طراحی منطق دامنه، تبادل ناهمگام پیغام، مدل اکتور، همروندی}

\pagenumbering{harfi}


\newpage
\thispagestyle{empty}
\mbox{}

\small{
\tableofcontents
%\listoftables
\listoffigures
}


\newpage


\pagestyle{fancy}
\fancyhead{} 
\fancyhead[RO]{\leftmark}
\fancyhead[LO]{\thepage}
\fancyhead[LE]{\rightmark}
\fancyhead[RE]{\thepage}
\fancyfoot{} 
\renewcommand{\headrulewidth}{0.6pt} 
\renewcommand{\footrulewidth}{0pt}

\pagenumbering{arabic}\setcounter{page}{1}
\setcounter{secnumdepth}{3}
\chapter{مقدمه}
\label{chapter:Introduction}
\thispagestyle{plain}
از دیدگاه مهندسی نیازمند‌ی نرم‌افزار\LTRfootnote{software requirement engineering}، مجموعه‌ی نیازمندی‌های یک سیستم نرم‌افزاری را می‌توان در دو دسته‌ی کلی {\it کارکردی\LTRfootnote{functional}} و {\it غیرکارکردی\LTRfootnote{non-functional}} طبقه‌بندی نمود \cite{Greene05}. مطابق این طبقه‌بندی، نیازهای کارکردی عملکرد درست نرم‌افزار را از دید کاربران تعیین می‌کنند. در دید غیرکارکردی نیز به چگونگی عملکرد سیستم و شرایط آن می‌پردازد و در واقع مشخص می‌کند که سیستم با چه کیفیتی در محیط اجرای خود فعالیت خواهد نمود. در ادامه‌ی این متن تنها به بررسی نیازهای کارکردی نرم‌افزار خواهیم پرداخت.

برای اطمینان از صحت پیاده‌سازی نیازمندی‌های نرم‌افزار در متن پیاده‌سازی شده روش‌های مختلفی پیشنهاد شده و مورد استفاده قرار می‌گیرد. برای مثال \emph{آزمون نرم‌افزار}\LTRfootnote{software testing}، دسته‌ای از روش‌ها را برای بازرسی \gls*{مطابقت}\LTRfootnote{conformance} متن برنامه\LTRfootnote{source code} و نیازهای سیستم معرفی می‌کند. در روش‌های آزمون از اِعمال \glspl*{مورد آزمون}\LTRfootnote{test cases} بر سیستم تحت آزمون برای سنجش این مطابقت استفاده می‌شود. هر مورد آزمون عبارت است از ترتیبی تعریف شده از عملیات بر روی سیستم و هم‌چنین نتیجه‌ای که در قبال این عملیات از سیستم مورد انتظار است \cite{ammann08}.

یک طبقه‌بندی شناخته شده، روش‌های آزمون نرم‌افزار را به دو دسته‌ی کلی \emph{آزمون کارکردی} و \emph{آزمون ساختاری} تقسیم می‌کند \cite{Jorgensen08}، اگرچه این تقسیم‌بندی کاملاً استاندارد و همه‌گیر نیست. آزمون کارکردی (که معمولاً با نام آزمون جعبه-سیاه\LTRfootnote{black-box testing} شناخته می‌شود) بدون اطلاع از نحوه‌ی پیاده‌سازی سیستم‌ انجام شده و تنها به توصیف‌های سطح بالای سیستم متکی است. در مقابل ایده‌ی آزمون کارکردی، آزمون ساختاری قرار دارد (که با نام آزمون جعبه-سفید\LTRfootnote{white-box testing} نیز شناخته‌ می‌شود). در این ایده، از ساختار داخلی کد و اطلاعات مربوط به نحوه‌ی پیاده‌سازی استفاده شده و تلاش می‌شود تمامی حالت‌هایی که درون متنِ برنامه‌ی پیاده شده است مورد بررسی قرار گیرد.

از جمله‌ی روش‌های جعبه‌-سیاه شناخته شده، \emph{آزمون مبتنی بر مدل}\LTRfootnote{model-based testing}\cite{Ap97modelbased} است. آزمون مبتنی بر مدل تلاش می‌کند تا با مدل‌سازی رفتار سیستم (که از آن  به عنوان توصیف\LTRfootnote{specification} سیستم یاد می‌شود) و تحلیل آن، به طور خودکار به تولید موارد آزمون پرداخته و علاوه بر آن مراحل اجرا و بررسی نتایج آزمون را نیز به صورت خودکار انجام دهد. به این ترتیب، در آزمون مبتنی بر مدل مهم‌ترین فعالیت در طراحی آزمون‌ها، ساختن مدلی از  رفتار سیستم و تعیین چگونگی ارتباط آن با سیستم اصلی است. با توجه به خودکار بودن مراحل تولید موارد آزمون، لازم است نمادگذاری‌های به کار رفته برای مدل‌سازی با اتکا بر \glspl*{روش صوری}\LTRfootnote{formal methods} طراحی شوند تا امکان پردازش خودکار آن‌ها فراهم شود.

استفاده از ایده‌ی آزمون مبتنی بر مدل خصوصاً در مورد سیستم‌های با توصیف پیچیده بسیار مفید خواهد بود. در این سیستم‌ها، به دلیل تعدّد عملکرد‌های سیستم، برای اطمینان از صحت کارکرد آن‌ها باید تعداد زیادی مورد آزمون طراحی نمود. واضح است که در صورت استفاده از روش‌های متداول آزمون، نگه‌داری این مجموعه‌ی آزمون دشوار خواهد بود، زیرا به وجود آمدن تغییراتی در متن برنامه ممکن است تغییرات زیادی در آزمون‌های طراحی شده را طلب کند. در مقابل در  روش‌های آزمون مبتنی بر مدل با بروز تغییرات در ساختار و متن برنامه‌ی تحت آزمون تنها کافی است مدل توصیف‌کننده سیستم، تغییر کند، در این صورت موارد آزمون مجدداً با استفاده از این مدل جدید به صورت خودکار تولید خواهند شد. مزیت دیگر استفاده از روش‌های آزمون مبتنی بر مدل، خصوصاً برای سیستم‌های با رفتارهای نسبتاً پیچیده، اطمینان از فراموش نشدن برخی آزمون‌های مهم است. در روش‌های متداول آزمون، به دلیل دستی بودن فرآیند طراحی آزمون‌ها و نیز تعدد آن‌ها، ممکن است برخی از آن‌ها (به خاطر احتمال بروز خطای انسانی) از قلم بیفتند که این احتمال در روش‌های آزمون مبتنی بر مدل (با توجه به سیستماتیک بودن مراحل تولید موارد آزمون) به طور کلی از بین می‌رود.

بسته به نمادگذاری به کار رفته در مدل‌سازی و نیز روش تولید موارد آزمون از روی این توصیف‌ها، روش‌های متفاوتی برای آزمون مبتنی بر مدل ارائه شده است. در این‌جا بر گونه‌ی خاصی از آزمون مبتنی بر مدل به نام \emph{آی‌او‌کو}\LTRfootnote{ioco}\cite{tret96} تمرکز می‌کنیم. در این روش از نمادگذاری \emph{ماشین‌های گذار}\LTRfootnote{transition systems} برای توصیف رفتار سیستم استفاده می‌شود. 

در گونه‌های امروزی آی‌او‌کو، موارد آزمون به صورت در-لحظه\LTRfootnote{on-the-fly} تولید می‌شوند. به عبارت دیگر آزمون‌گر مبتنی بر مدل در زمان اجرای آزمون‌ها و با توجه به الگوریتم، یک مورد آزمون تولید و بر روی سیستم اعمال می‌کند. هم‌چنین، آزمون‌گر خروجی‌های سیستم را مشاهده و صحت آن را بررسی می‌کند. یکی از دلایل استفاده از روش‌های در-لحظه برای تولید موارد آزمون، احتمال مشاهده‌ی رفتارهای غیرقطعی\LTRfootnote{non-deterministic} از سیستم است، به این معنی که ممکن است سیستم به ازای یک ورودی ثابت امکان تولید چندین خروجی مختلف و مجاز را داشته باشد. به این ترتیب آزمون‌گر می‌تواند در زمان تولید موارد آزمون به طور پویا و بر اساس نحوه‌ی رفتار سیستم نسبت به تولید ورودی مناسب برای سیستم اقدام نماید.

\section{انگیزه‌ی پژوهش}
روش‌های آزمون مبتنی بر مدل کاستی‌هایی نیز دارند. برای مثال در آی‌اوکو، استفاده از ماشین گذار برای توصیف سیستم تحت آزمون می‌تواند منجر به تولید مدل‌های بسیار پیچیده‌ای شود، زیرا ماشین‌های گذار علی‌رغم قدرت بیان بالا، چنا‌ن‌چه خواهیم دید، از نظر توصیفی نمادگذاری سطح پایینی محسوب می‌شوند و بنابراین مدل‌سازی جزئیات سیستم ممکن است حجم زیادی از پیچیدگی را در مدل‌ها به وجود آورد.

یکی از مهم‌ترین عوامل ایجاد کننده‌ی پیچیدگی در مدل‌های مورد استفاده در آی‌اوکو، نیاز به مدل‌سازی مقادیر داده‌ای در سیستم‌هایی است که از پارامترهای داده‌ای به همراه ورودی‌‌ها و خروجی‌های خود استفاده می‌کنند. پیچیدگی مدل‌های تولید شده در صورت وجود پارامترهای داده‌ای، با تعداد پارامترها و مجموعه‌ی مقادیری که هر پارامتر می‌تواند (و یا لازم است) به خود بگیرد به طور مستقیم در ارتباط است. البته برای کاستن از این پیچیدگی، راه‌کارهایی نیز ارائه شده است. برای مثال چنان‌چه خواهیم دید، گسترشی از رابطه‌ی آی‌او‌کو با مقادیر داده‌ای ارائه شده است که امکان مدل‌سازی پارامترهای داده‌ای را فراهم می‌سازد. 

کاستی‌های روش آی‌‌اوکو منحصر به این موارد نیست. برای مثال در این روش معیاری برای مقایسه‌ی موارد آزمون مختلف وجود ندارد. بنابراین ممکن است برخی موارد آزمون، بدیهی (مثلاً فقط شامل یک ارسال داده به سیستم) و برخی دیگر بسیار پیچیده (مثلاً شامل ده‌ها و یا صدها رفتار متوالی با سیستم) باشند و یا حتی برخی همدیگر را شامل شوند. مثال دیگر، اتکای روش آی‌او‌کو بر تولید آزمون‌ها به صورت کاملاً برخط است. استفاده از چنین روشی همواره مطلوب نیست زیرا ممکن است تولید موارد آزمون (به دلیل بزرگ بودن مدل آن) هزینه‌ی زیادی در بر داشته باشد که تکرار این مسئله احتمالاً مطلوب نیست. البته در این پژوهش به این موارد به طور مستقیم پرداخته نخواهد شد، اما چنان‌چه خواهیم دید حاصل این پژوهش می‌تواند راه را برای رفع چنین مشکلاتی هموار نماید.

\section{صورت مسئله}
تمرکز اصلی این پژوهش بر آزمون مبتنی بر مدل سیستم‌های وابسته به داده\LTRfootnote{data dependent} قرار دارد. سیستم‌های وابسته به داده معمولاً حجم زیادی از اطلاعات را با محیط خود مبادله‌ می‌‌کنند و رفتار آن‌ها وابسته به محاسباتی است که بر روی مقادیر داده‌ای انجام می‌دهند. 

چنان‌چه پیش از این نیز گفته شد، در حال حاضر گسترشی از روش آی‌اوکو به هدف مدل‌سازی و آزمون این سیستم‌ها معرفی شده‌ است. با این حال این روش نیز از چند جنبه دچار ضعف است. این پژوهش برای یافتن راه‌کاری جامع برای دو مورد از اساسی‌ترین مشکلات این روش تلاش می‌کند. این دو مورد را می‌توان به ترتیب زیر برشمرد:
\begin{itemize}
\item این روش در مورد مقادیر داده‌ای که برای آزمون مورد استفاده قرار می‌گیرند اظهار نظر نمی‌کند. به عبارت دیگر در این روش‌ راه‌کاری برای تعیین این‌که چه مقادیر داد‌ه‌ای باید در یک سناریوی رفتاری تولید شده (از روی مدل ماشین‌گذار) به سیستم داده شود، ارائه نشده است. به همین سبب در این روش‌ لازم است تا تمامی مقادیر داده‌ای ممکن بر روی سیستم آزمایش شود که به نظر منطقی نمی‌رسد.

از طرف دیگر مطالعه بر روی ادبیات حوزه‌ی آزمون نرم‌افزار نشان می‌دهد که در گذشته پژوهش‌های مستقلی بر روی روش انتخاب مؤثر داده‌های آزمون انجام شده است و این پژوهش‌ها منجر به ارائه‌ی روش‌های متعددی برای شناسایی و  دسته‌بندی مقادیر داده‌ای به اهداف آزمون نرم‌افزار شده است. یکی از این روش‌ها، روش \emph{افراز رده‌ای}\LTRfootnote{category partition method} است. چنان‌که خواهیم دید این روش به طور سطح بالا نشان می‌دهد که چگونه می‌توان دامنه‌ی مقادیر پارامترهایی که بین سیستم و محیط مبادله می‌شوند را به قسمت‌های مناسبی شکست و از بین آن‌ها تنها برخی از مقادیر را برگزید.

با توجه به این موارد، اولین جزء از صورت مسئله‌ی این پژوهش را می‌توان به صورت استفاده هم‌زمان از روش افراز رده‌ای در کنار آی‌اوکو برای تولید موارد آزمون تعریف نمود. در این قسمت تلاش شده است تا این استفاده‌ی هم‌زمان در قالب یک چهارچوب یک‌پارچه صورت گیرد. به این معنی که در چهارچوب حاصل، این دو روش هم از نظر نحو و نمادگذاری که برای توصیف سیستم مورد استفاده قرار می‌دهند و هم از نظر معنایی به صورت کاملاً \gls*{یک‌پارچه} و \gls*{بدون درز}\LTRfootnote{seamless} به نظر آیند. 

\item گسترش داده‌ای آی‌اوکو پشتیبانی ابزاری چندانی ندارد و در حال حاضر هیچ ابزار آزمون‌گری به طور کامل از آن پشتیبانی نمی‌کند. چهارچوب حاصل برای آزمون مبتنی بر مدل در این پژوهش به عنوان مبنایی برای تولید یک مجموعه‌ی ابزار آزمون‌گر کامل، مورد استفاده قرار گرفته است. برای پیاده‌سازی این ابزار از تجربه‌ی پیاده‌سازی آزمون‌گرهای مبتنی بر مدل و هم‌چنین آزمون‌گرهای دیگری که پشتیبانی گسترده‌ای از آزمون مقادیر داده‌ای می‌کنند استفاده شده است.
\end{itemize}

ادامه‌ی این متن به طور مشروح به روش رسیدن به این اهداف خواهد پرداخت. با این حال قبل از ورود به بحث، برخی از مهم‌ترین دستاوردهای این پژوهش را به طور فهرست‌وار ارائه می‌کنیم.
\section{روش پژوهش}
ارزیابی عملی با مطالعه موردی
\section{روش ارزیابی}
GQM
\section{خلاصه‌ی دستاوردهای پژوهش}
برخی از دستاوردهای این پژوهش را می‌توان به این ترتیب برشمرد:
\begin{itemize}
\item \textbf{ارائه‌ی راه‌کاری ابتدایی برای طراحی توصیف‌های سیستم تحت آزمون:} کار طراحی توصیف سیستم (چه برای مدل‌های رفتاری و چه برای مدل‌های داده‌ای) باید با استفاده از دانش به دست آمده در فرآیند تولید نرم‌افزار و هم‌چنین اطلاعات مربوط به نحوه‌ی پیاده‌سازی متن برنامه صورت گیرد. در این پژوهش روشی سطح بالا برای استخراج توصیف‌های سیستم با توجه به این موارد ارائه شده است. این روش امکان تولید همزمان مدل‌های رفتاری و داده‌ها را داراست.

\item \textbf{ارائه‌ی نمادگذاری یک‌پارچه برای توصیف مدل‌ها:} برای حفظ یک‌پارچگی توصیف‌های رفتاری و داده‌ای، تمامی مدل‌های تولید شده در این چهارچوب در زبان یو‌ام‌ال\LTRfootnote{Unified Modeling Language (UML)} طراحی شده‌اند. چنان‌چه خواهیم دید در این چهارچوب تلاش شده است تا حد امکان به استانداردهای نحوی و معنایی یو‌ام‌ال وفادار بمانیم.

\item \textbf{امکان توصیف محدودیت‌های محیط و سناریوهای آزمون:} در روش پیشنهادی این پژوهش، می‌توان یک توصیف ارائه شده از سیستم را چندین بار و در شرایط محیطی متفاوت آزمود. مطابق تعریف، شرایط محیطی در واقع نشان‌دهنده‌ی امکان و یا عدم امکان بروز برخی از رفتارها (که در سیستم پیاده‌سازی شده است) از سیستم است. چنین امکانی مخصوصاً در شرایطی مفید است که سیستم عملکرد‌های مختلفی داشته باشد و امکان تأثیر این عملکردها بر نحوه‌ی اجرای یک‌دیگر نیز ممکن باشد.

\item \textbf{نگارش ابزار آزمون:} با توجه به نو بودن بسیاری از ایده‌های این پژوهش، ابزاری جدید به منظور بررسی صحت ایده‌های این کار و هم‌چنین استفاده‌های بعدی، طراحی و پیاده‌سازی شده است. این آزمون‌گر علاوه بر پشتیبانی کامل از آی‌اوکو (و گسترش آن به همراه داده) از تمامی ایده‌های مطرح شده در این پژوهش نیز پشتیبانی می‌کند.  این ابزار به طور کامل در زبان جاوا\LTRfootnote{Java} پیاده‌سازی شده است. چنان‌چه در فصل \ref{section:fwimpl} خواهیم دید علاوه بر پیاده‌سازی آزمون‌گر، از تعدادی بسته‌‌ی متن‌باز برای اهدافی مانند طراحی مدل‌ها نیز استفاده شده است.

\item \textbf{مطالعه‌ی موردی بر روی یک سیستم نمونه:} در نهایت، نمونه‌ی یک سیستم واقعی با استفاده از روش معرفی شده در این پژوهش و هم‌چنین ابزار طراحی شده بر پایه‌ی آن، مورد آزمون قرار گرفته است. از نتایج این مطالعه برای تحلیل کارکرد این چهارچوب در مقابل روش‌های عادی آزمون مبتنی بر مدل (خصوصاً آی‌او‌کو) استفاده شده است. به این منظور معیاری عددی (مبتنی بر معیار پوشش متن\LTRfootnote{code coverage} برنامه) برای مقایسه‌ی آزمون‌های تولید شده توسط روش‌های مختلف تعریف شده است.
\end{itemize}
مجموعه‌ی دستاوردهای این پژوهش را می‌توان از دید تاثیری که بر ساختاری داخلی برنامه‌ی آزمون‌گر دارند نیز مورد بررسی قرار داد. به این منظور نمای سطح بالا برای تولید آزمون‌ها و بررسی نتایج در روش اصلی آزمون مبتنی برمدل در شکل \ref{fig:mbtOverview} آمده است. این ساختار در آزمون‌گر طراحی شده در این پژوهش نیز عیناً وجود دارد با این تفاوت که \textit{اولاً} روش توصیف سیستم تحت آزمون به طور کلی تغییر کرده و \textit{ثانیاً} روش تولید ورودی مجاز سیستم مجدداً تعریف شده است. واضح است که اجزای دیگر آزمون‌گر نیز ممکن است تحت تاثیر این تغییرات، تغییر کنند با این حال موارد ذکر شده مهم‌ترین مواردی است که در این پژوهش مورد طراحی مجدد قرار می‌گیرند. 

\begin{figure}
    \begin{center}
    \begin{tabular}{| c |}
   	\hline
	\\
  	 \includegraphics[width=11cm]{2-Preliminaries/Figures/mbtOverview.pdf}
	\\
	\hline
  \end{tabular}
  \end{center}
   \caption{\label{fig:mbtOverview} نمای سطح بالای مراحل آزمون در روش مبتنی بر مدل \cite{utting06}}
\end{figure}


\section{ساختار پایان‌نامه}
برای بررسی این موارد، ساختار این متن در ۶ فصل تنظیم گردیده است:
\begin{strict_itemize}
\item
 فصل \ref{chapter:Preliminaries} به بررسی برخی پیش‌نیازهای تعریف چهارچوب پیشنهادی می‌پردازد. رابطه‌ی مطابقت آی‌اوکو و روش افراز رده‌ای در این بخش به طور خلاصه مورد بررسی قرار گرفته‌اند.
\item
در فصل \ref{chapter:RelatedWork} برخی از کارهای قبلی و مرتبط با مسئله‌ی آزمون مبتنی بر مدل و راه‌حل‌های آن معرفی شده‌اند. هم‌چنین بعضی از مهم‌ترین ابزارهای آزمون نرم‌افزار که مبتنی بر مدل نیستند اما پشتیبانی خوبی برای آزمون سیستم‌های مبتنی بر داده دارند نیز بررسی شده‌اند. 
\item
فصل \ref{chapter:proposedFramework} ایده‌ی پیشنهادی در این پژوهش به طور مبسوط مورد بحث قرار داده است. این ایده (آن‌چنان که پیش از این نیز اشاره شد) عبارت از ارائه‌ی یک چهارچوب برای آزمون مبتنی بر مدل سیستم‌هایی است که هم از حیث رفتاری و هم از تاثیرپذیری از مقادیر داده‌ای که مبادله می‌کنند، پیچیده‌اند. در این فصل هم به نحو و هم به معنای این چهارچوب پرداخته شده است. برای نحو چهارچوب پیشنهادی از زبان یو‌ام‌ال استفاده شده است و معنای آن نیز بر روش آی‌اوکو (که در فصل \ref{chapter:Preliminaries} مورد بررسی قرار می‌گیرد) استوار است.
\item
 در فصل \ref{chapter:caseStudy} یک مطالعه‌ی موردی از آزمون یک سیستم واقعی از حوزه‌ی نرم‌افزارهای مالی، با استفاده از روش پیشنهادی، مورد بررسی قرار خواهد گرفت. برای این منظور معیاری عددی برای مقایسه‌ی آزمون‌های تولید شده توسط این چهارچوب و آزمون‌های تولید شده توسط روش‌های دیگر ارائه شده و نتایج حاصله تحلیل شده‌اند.
\item
نهایتاً فصل \ref{chapter:Conclusion} برخی نکات پایانی را مطرح و در مورد جهت‌گیری‌های احتمالی آینده‌ی این پژوهش مطالبی ارائه می‌کند.
\end{strict_itemize}
\chapter{پیش‌زمینه تحقیق}
\label{chapter:Preliminaries}
\thispagestyle{plain}
در این فصل به طور اجمالی مروری بر پیش‌زمینه‌ی پژوهش انجام شده است. در هر بخش سعی شده است که با حفظ اختصار، تنها جنبه‌های  کاربردی مرتبط با پژوهش مطرح گردد.

\section{مدل بازیگر}

\gls{مدل بازیگر}%
، که توسط هیوئیت و آقا \cite{Hewitt1972,Agha1987,Agha1990} ایجاد شده‌است، یک نمایش سطح بالا از سیستم‌های توزیع‌شده فراهم می‌کند. 
\gls{بازیگر}ها
اشیای \gls{لفافه‌بندی‌شده}‌ای هستند که به صورت \gls{همروند} فعالیت می‌کنند و دارای \gls{رفتار}\LTRfootnote{Behavior} قابل تغییر هستند. 
بازیگرها \gls{حالت  مشترک}\LTRfootnote{Shared State} ندارند و تنها راه ارتباط بین آنها تبادل ناهمگام پیغام است. 
 در مدل اکتور فرضی در مورد مسیر پیغام و میزان تاخیر در رسیدن پیغام وجود ندارد، در نتیجه ترتیب رسیدن پیغام‌ها \gls{غیرقطعی} است.
 
 در یک دیدگاه می‌توان بازیگر را یک \gls{شی‌ء} در نظر گرفت که به یک ریسمان\gls{ریسمان}\LTRfootnote{ُThread} کنترل، یک صندوق پست و یک نام غیر قابل تغییر و به صورت سرارسی یکتا \LTRfootnote{Globally Unique} در نظر گرفت. برای ارسال پیغام به یک بازیگر، از نام آن استفاده می‌شود. در این مدل، نام  یک بازیگر را می‌توان در قالب پیغام  ارسال کرد.
 
 یک بازیگر در نتیجه‌ی دریافت پیغام احتمالا محاسباتی انجام می‌دهد و در نتیجه‌ی آن یک از ۳ عمل زیر را انجام می‌دهد:
\begin{itemize}
\item[ارسال پیغام]
\item[ایجاد بازیگر جدید]
\item[تغییر حالت محلی]

\end{itemize} 
از نظر \gls{معناشناسی}LTRfootnote{Semantics} مشخصات کلیدی مدل اکتور عبارتند از: لفافه‌بندی حالت،  زمانبندی منصفانه‌ی بازیگرها و تبادلات پیغام‌ها، و نامگذاری جهانی\LTRfootnote{universal naming}
 
پاسخگویی به هر پیام شامل برداشتن آن پیام از صندوق پستی و اجرای عملیات متناسب با آن است.
این اجرای عملیات به صورت \gls{تجزیه‌ناپذیر}\LTRfootnote{Atomic} و بی‌وقفه خواهد بود.

همان گونه که گفته‌شد، مدل بازیگر سیستم را در سطح بالایی از انتزاع مدل می‌کند.
این ویژگی دامنهٔ سیستم‌های قابل مدلسازی توسط مدل بازیگر را بسیار وسیع نموده‌است.
انواع سیستم‌های سخت‌افزاری و نرم‌افزاری طراحی‌شده برای زیرساخت‌های خاص یا عام، و همچنین الگوریتم‌ها و پروتکل‌های توزیع‌شدهٔ مورد استفاده در شبکه‌های ارتباطی از جملهٔ موارد مناسب برای بهره‌گیری از مدل بازیگر هستند.

تأکید ویژهٔ مدل بازیگر بر ارسال ناهمگام پیام نیز یکی از ویژگی‌های این مدل است که آن را برای مدلسازی سیستم‌های توزیع‌شده مناسب کرده‌است.
این مکانیزم ارسال پیام می‌تواند برای مدلسازی ارتباطات شبکه‌ای و تیادل اطلاعات در محیط‌های توزیع‌شده مورد استفاده قرار گیرد.
در حقیقت مدل بازیگر با توجه به اینکه انتزاعی سطح بالا از همهٔ مؤلفه‌های موجود در سیستم‌های توزیع‌شده است، بسیار نزدیک به واقعیت است و در مواردی که سیستم مورد نر یک سیستم توزیع‌شده با مؤلفه‌های موازی باشد کار مدلسازی را بسیار تسهیل می‌کند.




\subsection{جایگاه آزمون مبتنی بر مدل در میان روش‌های بازرسی کارکرد نرم‌افزار}
روش‌های زیادی برای بازرسی سیستم‌های نرم‌افزاری به هدف اطمینان از کارکرد صحیح آن‌ها ارائه شده است. اگرچه نمی‌توان به طور مشخص در مورد برتری یکی از این روش‌ها به دیگری نظر داد اما می‌‌توان آن‌ها را از نظر شیوه‌ی استفاده طبقه‌بندی نمود. در زیر، یک طبقه‌بندی برای روش‌های آزمون مبتنی بر مدل، با توجه به نوع محصولات مورد استفاده و روش استفاده از آن‌ها، آمده است.

\begin{description}
\item[روش‌های مبتنی بر بررسی مدل:]
بالای بازرسی مدل‌ها به دلیل نیاز به ساخت فضاهای حالت به طور کامل و پیمایش آن‌هاست.
\item[روش‌های ]\LTRfootnote{static analysis} در این روش‌ها خصوصیات مورد نظاده از این روش‌ها را به عنوان جایگزین 
\end{description}

آزمون مبتنی بر مدل با بهره‌گیری از ایده‌ی مدل‌سازی از روش‌های تحلیل مدل از یک سو و اجرای موارد آزمون تولید شده از سوی دیگر تلاش می‌کند گونه‌ای کارا از آزمون نرم‌افزار را ارائه نماید. آزمون مبتنی بر مدل با ادر ادامه‌ی متن حاضر برای راحتی سیستم نامیده خواهد شد) اجرا می‌شوند. 

\subsection{؟؟}
\label{subsection:ioco}
رابطه مطابقت موجود و اعلام نظر در مورد مطابقت و یا عدم مطابقت این دو ارائه می‌کند.


\section{طراحی مبتنی بر دامنه}
\label{section:CategoryPartitionMethod}
اگرچه رابطه‌ی معرفی شده از مطابقت ماشین‌های گذار نمادین امکان بازرسی رفتار سیستم به همراه داده را فراهم می‌کند، با این‌حال این روش مشخص نمی‌کند که مقادیر داده‌ای که باید مورد آزمون قرار بگیرند از چه طریقی باید مشخص شوند. 
 
بعد از طی این مراحل و تکمیل اطلاعات در مورد مقادیر داده‌ای، در نهایت موارد آزمون نهایی با جایگزینی گزینه‌های انتخاب شده از رده‌های مختلف و اجرای رفتار مشخص شده‌ی آزمون برای هر واحد کارکردی، تولید می‌شود.

\subsubsection{؟}
اگرچه روش افراز داده‌ای راه‌حلی سطح بالا برای رده‌بندی داده‌های آزمون به دست می‌دهد، در این روش تمامی گزینه‌های انتخاب شده باید در یک سطح بیان شوند و امکان طبقه‌بندی داده‌ای در آن وجود ندارد (به این معنی که نمی‌توان یک رده‌ی انتخاب شد

\chapter{پژوهش‌های مرتبط}
\label{chapter:RelatedWork}
\thispagestyle{plain}
در این فصل به ارائه‌ی برخی کارهای پیشین و مرتبط به موضوع این پژوهش خواهیم پرداخت. در مورد هر یک از این موارد به ارتباط آن با بحث جاری، کاربرد و یا نقاط تأثیرگذار آن در موضوع این پژوهش و هم‌چنین ضعف ها و نقایص آن‌ها پرداخته شده است. 
\section{الگوهای برنامه‌نویسی بازیگر}
\label{section:actorPatterns}
در برنامه‌نویسی همروند با بازیگر‌ها دو نوع الگوی کلی معرفی شده است \cite{Agha1990}: یکی  \textit{\gls{تقسیم-و-حل}}\LTRfootnote{devide and conquer} و دیگری \textit{\gls{خط لوله}}\LTRfootnote{pipeline}. 
در روش تقسیم-و-حل مسئله‌ی مورد بحث به زیربخش‌های کوچکتر و مستقل تقسیم می‌شود که هرکدام به صورت مستقل حل می‌شوند و نتایج هر زیربخش برای نتیجه‌گیری کلی ادغام می‌شوند. در برنامه‌نویسی به مدل بازیگر، برای پیاده‌سازی این الگو یک بازیگر رئیس\LTRfootnote{master} در نظر گرفته می‌شود که تعدادی بازیگر کارگر\LTRfootnote{worker} را برای حل زیربخش‌های مسئله ایجاد می‌کند. عمل تقسیم به وسیله‌ی فرستادن پیغام‌ حاوی حالت لازم برای حل زیر بخش به کارگر‌ها انجام می‌شود. کارگرها به نوبه‌ی خود منطق لازم برای حل زیر بخش را ایجاد نموده و نتیجه را به صورت پیغام دیگری برای بازیگر رئیس ارسال می‌کنند. نهایتا رئیس با ادغام نتایج جواب نهایی 
مسئله را تولید می‌کند. شایان ذکر است که فازهای تقسیم و حل لزوما توسط بازیگر یکسان اجرا نمی‌شوند. ممکن است اجرای فاز حل به بازیگر دیگری سپرده شود.\cite{Feng08scalablemodels}
مثال دیگری از پیاده‌سازی الگوی تقسیم-و-حل در مدل بازیگر  در \cite{Feng08scalablemodels} آمده است که در آن الگوریتم جستجوی سریع\LTRfootnote{quick sort} توسط این الگو پیاده شده است.
\begin{figure*}
    \begin{center}
	\includegraphics[width=16cm]{3-RelatedWork/Figures/Divide_and_Conquer.pdf}
    \end{center}
    \caption{\label{fig:divide_conquer}  شمای کلی از الگوی تقسیم-و-حل در مدل بازیگر }
\end{figure*}
شکل \ref{fig:divide_conquer} شمایی از نحوه‌ی پیاده‌سازی الگوی تقسیم-و-حل در مدل بازیگر را نمایش می‌دهد.\\
الگوی خط لوله برای حالت‌هایی مناسب است که فعالیت قابل تقسیم به بخش‌های افزایشی باشد. در این صورت هر بازیگر تغییرات مربوطه را در مدل ایجاد می‌کند و آن را به عنوان پیغام به بازیگر بعدی در خط لوله منتقل می‌کند.

\begin{figure*}
    \begin{center}
	\includegraphics[width=16cm]{3-RelatedWork/Figures/pipeline.pdf}
    \end{center}
    \caption{\label{fig:pipeline}  مثالی از الگوی خط لوله (پردازش تصویر) }
\end{figure*}
به عنوان مثالی از الگوی خط لوله یک برنامه‌ی پردازش تصویر را در نظر بگیرید. هر مرحله از خط لوله، تغییراتی را در تصویر دریافتی ایجاد می‌کند و تصویر نتیجه را به مرحله‌ی بعد منتقل می‌کند. در پیاده‌سازی با روش بازیگر، هر مرحله به صورت یک بازیگر مدل می‌شود و تصویر به صورت پیغام بین مراحل رد و بدل می‌شود. در شکل \ref{fig:pipeline} شمایی از این الگو نشان داده شده‌ است. \\
در پژوهش‌های انجام شده مشخص شد که الگوهای ارائه شده صرفا الگوهای کلی همروندی هستند و جزئیات این الگوها در طراحی منطق دامنه، نحوه‌ی طراحی پیغام‌ها بررسی نشده اند .

\section{ همگام‌سازی و هماهنگی بازیگرها }
\label{section:coordinationAndSyncronization}
همان‌طور که در بخش‌های قبل ذکر شد،  مدل بازیگر دارای خاصیت ناهمگامی‌ است و ترتیب پیغام‌هایی که یک بازیگر دریافت می‌کند وابسته به ترتیب فرستاده شدن پیغام‌ها نیست. نتیجه‌ی این خاصیت این است که تعداد ترتیب\LTRfootnote{ordering}‌های دریافت پیغام‌ها در مدل بازیگر نمایی است\cite{KarmaniAgha_Actors_11}. به دلیل اینکه فرستنده‌ی پیغام از حالت محلی بازیگر گیرنده اطلاعی ندارد، ممکن است بعضی از ترتیب‌های ذکر شده برای پیغام‌ها مطلوب نباشد. به عنوان مثال الگوریتمی را در نظر بگیرید که زیر بخش‌های مختلف آن به بازیگر‌هایی فرستاده شده و نتایج آن دریافت می‌شود ولی در آن ترتیب دریافت نتایج اهمیت داشته باشد.  نیاز به این نوع اولویت‌بندی‌ها در مدل بازیگر منجر به ایجاد پیچیدگی در محاسبات همروند می‌شود و در صورت پیاده‌سازی نامناسب باعث ایجاد ناکارامدی در برنامه‌ها می‌شود. راه حل این مسئله در مدل اکتور همگام‌سازی است. در مدل بازیگر، بازیگر‌ها برای همگام‌سازی باهم ارتباط برقرار می‌کنند. در این قسمت دو نوع الگوی هماهنگی بازیگر‌ها را معرفی می‌کنیم: تبادل پیغام شبه آرپی‌سی (فراخوانی رویه راه دور)\LTRfootnote{Remote Procedure Call} و قیود همگام‌سازی محلی \LTRfootnote{Local Synchronization Constraints}. \cite{Agha1990,Agha93abstractionand,Papaioannou,KarmaniAgha_Actors_11} 
\subsection{تبادل پیغام شبه-آرپی‌سی}
در ارتباط شبه‌-آرپی‌سی، فرستنده‌ پس از ارسال پیغام منتظر گرفتن پیغام پاسخ از طرف گیرنده می‌ماند. رفتار بازیگر در این مدل به ترتیب زیر است:
\begin{enumerate}
\item بازیگر فرستنده درخواست را در قالب یک پیغام به بازیگر گیرنده ارسال می‌کند.
\item سپس فرستنده صندوق پیغام‌ها را بررسی می‌کند.   
\item اگر پیغام بعدی پاسخ درخواست ارسال شده باشد اقدام مناسب صورت می‌گیرد و فعالیت بازیگر ادامه پیدا می‌کند.
\item اگر پیغام بعدی پاسخ درخواست ارسال شده نباشد پیغام جاری در صورت امکان (بسته به منطق برنامه) پردازش می‌شود و در غیر این صورت برای پردازش در آینده به صندوق پیغام‌ها برگردانده می‌شود.
\end{enumerate}
شکل \ref{fig:rpc} مثالی از  پیاده‌سازی ارتباط شبه-آرپی‌سی در مدل بازیگر را نشان می‌دهد. ارتباط شبه-آرسی‌پی در دو نوع سناریوی خاص مفید و ضروری است: یک سناریو این است که بازیگر نیاز به ارسال پیغام به صورت ترتیبی به یک یا چند بازیگر خاص دارد و تا حاصل شدن اطمینان از رسیدن پیغام قبلی پیغام بعد را ارسال نمی‌کند. سناریوی دوم این است که حالت\LTRfootnote{state} بازیگر فرستنده بستگی به محتوای پاسخ دارد. در این حالت بازیگر قبل از دریافت پاسخ مورد نظر، نمی‌تواند پیغام‌های بعدی را به درستی پردازش کند. نکته‌ی قابل توجه این است که با توجه به شباهت ارسال پیغام شبه-آرپی‌سی به فراخوانی رویه\LTRfootnote{procedure}‌ها در زبان‌های \gls{ترتیبی}\LTRfootnote{sequential}، معمولا برنامه‌نویسان گرایش به استفاده‌ی بیش از حد از این نوع تبادل پیغام دارند که این ممکن است با ایجاد وابستگی‌های بی‌مورد در اشیاء برنامه، علاوه بر کاهش کارایی، منجر به ایجاد \gls{بن‌باز}\LTRfootnote{live lock} در برنامه شود (حالتی که یک بازیگر به علت انتظار برای پاسخی که هرگز دریافت نخواهد کرد، از پیغام‌های جدید مرتباً چشم‌پوشی می‌کند یا پردازش آنها را به تأخیر می‌اندازد).\\
امکان تبادل پیغام شبه-آرپی‌سی تقریبا در تمامی پیاده‌سازی‌های مدل بازیگر به صورت امکانات سطح زبان وجود دارد\cite{ActorsJVM2009}.

\begin{figure*}
    \begin{center}
	\includegraphics[width=10cm]{3-RelatedWork/Figures/RPC.pdf}
    \end{center}
    \caption{\label{fig:rpc} مثالی از ارتباط شبه-آرپی‌سی در بازیگرها) }
\end{figure*}
 
\subsection{قیود همگام‌سازی محلی}
استفاده از قیود همگام‌سازی محلی روشی برای اولیت‌بندی پردازش پیغام‌ها در مدل بازیگر است\cite{FrolundCoord}. برای توضیح مفهوم همگام‌سازی محلی مثالی در شکل \ref{fig:lsc} ارائه شده است. در این مثال بازیگر فایل پس از دریافت پیغام باز کردن فایل\LTRfootnote{open}، با استفاده از قیود همگام‌سازی خود را محدود به پردازش پیغام‌های \textit{بستن} , \textit{خواندن} می‌کند. در صورت عدم وجود امکانات مناسب برای قیود همگام‌سازی، برنامه‌نویس ناگزیر خواهد بود تا در میان منطق اجرای پیغام‌ها، میانگیر صندوق پیغام‌ها را بررسی و ترکیب یا ترتیب آنها را تغییر داده و یا با جستجو در آنها پیغام مناسب را انتخاب کند. این امر موجب مخلوط شدن منطق چگونگی پردازش پیغام  (چگونه) با منطق زمانی انتخاب پیغام (چه زمانی) می‌شود که در اصول نرم‌افزار پدیده‌ی نامطلوبی به حساب می‌آید\cite{KarmaniAgha_Actors_11}. به همین دلیل بسیاری از زبان‌ها و چارچوب‌های مبتنی بر بازیگر امکانات مناسبی برای پشتیبانی از قیود‌ همگام‌سازی محلی ارائه داده‌اند. به عنوان مثال در کتابخانه‌ی بازیگر اسکالا که در بخش \ref{section:scalaActorLib} معرفی شد، از مکانیزم تطابق الگو\LTRfootnote{pattern matching} برای اولیت بندی پردازش پیغام‌ها بدون اینکه با منطق اجرایی برنامه مخلوط گردد استفاده می‌شود.



\begin{figure*}
    \begin{center}
	\includegraphics[width=10cm]{3-RelatedWork/Figures/LSC.pdf}
    \end{center}
    \caption{\label{fig:lsc} مثالی از قیود همگام‌سازی محلی. بازیگر فایل به وسیله‌ی قیود همگام‌سازی محدود شده است. فلش عمودی به معنی ترتیب زمانی و برچسب‌های داخل دایره به معنی پیغام‌های قابل پردازش در هر حالت هستند. ) }
\end{figure*}

\chapter{طراحی بر اساس تبادل ناهمگام پیغام}
\label{chapter:proposedFramework}
\thispagestyle{plain}
در فصول گذشته روش آزمون ...

\section{معرفی مطالعه‌ی موردی }
\label{section:caseDesign}
\begin{itemize}
\item خصوصیت ۱

\item خصوصیت ۲
\end{itemize}

\subsection{زیر بخش}
 این فصل فرض بر آن است که موارد کاربرد برای سیستم مورد نظر تهیه شده و موجود است. 

\subsubsection{تولید مدل رفتاری سیستم}
در مدلی که در این مرحله تولید می‌شود، نیازی به مشخص کردن کنش‌‌هایی که باعث حرکت بین حالت‌های مختلف ماشین می‌شوند وجود ندارد. در واقع نمودار به دست آمده در این مرحله، تنها به هدف مدل‌سازی نمای سطح بالای عملکرد سیستم و مجموعه‌ی حالت‌های آن طراحی می‌ش

\subsubsection{تولید مدل رفتاری محیط}
آن‌چه که در ادامه‌ی این متن \emph{محیط عملکرد} و یا اختصاراً محیط نامیده می‌شود، به طور دقیق عبارت است از اکتورهایی که در یک مورد کاربرد با سیستم در ارتباطند. همان‌طور که پیش از این نیز گفته شد، برای حفظ هم‌خوانی ماشین‌های طراحی شد نمودار رفتار سیستم) نمی‌شود، بنابراین این مسیر هرگز اجرا نخواهد ش
\subsubsection{مشخص کردن و تعریف گونه‌های داده‌ای}
اگرچه این بخش به توصیف مدل‌سازی رفتاری سیستم و محیط اختصاص دارد، اما باید توجه کرد که تکمیل مدل‌های رفتاری نمی‌تواند کاملاً مستقل از داده‌هایی که بین اجزای مدل مبادله می‌شوند، انجام شود. همان‌طور که پیش از این  اصلی است. در چهارچوب پیشنهادی اجازه‌ی تعریف گونه‌های مستقل (یعنی گونه‌ای که از گونه‌ی دیگری گسترش نیافته است) وجو
\section{طراحی سیستم به روش ناهمگام}

\section{الگوها و سبک‌های طراحی}
\label{section:extracted_patterns}
پیش از این، نحو نمادگذاری مربوط به چهارچوب پیشنهادی این پژوهش تشریح شد. در این بخش، در مورد معنای هر یک از اجزای این نمادگذاری بحث خواهد شد. علاوه بر این، مطابقت معنایی این نمادگذاری را با مفاهیم آی‌اوکو مورد بررسی قرار داده و سپس الگوریتم جامعی برای تولید و اجرای یک‌پارچه‌ی آزمون‌هایی که با این نمادگذاری توصیف شده‌اند، ارائه خواهد شد.

\subsection{روشهای coordination }

\subsubsection{روش یک}
همان‌طور که پیش از این اشاره شد،‌ در چهارچوب پیشنهادی، توصیف‌های رفتاری چه برای سیستم و چه برای محیط در چندین نمودار حالت بیان می‌شود که هدف از آن کاهش پیچیدگی در طراحی مدل‌هاست. بنا به قرارداد، روش ترکیب این ماشین‌های حالت، روش میان‌گذاری است. به این معنی که 
\subsubsection{روش ۲}
همان‌طور که پیش از این اشاره شد، توصیف‌های رفتاری محیط نقش محدود‌کننده را در تولید موارد آزمون ایفا می‌کند. به عبارت دقیق‌



\subsection{سبک های طراحی}
سناریو‌های آزمون در چهارچوب پیشنهادی، تعیین کننده‌‌ی رو
با توجه به این تعاریف



\section{پیاده‌سازی}
\label{section:asyncImpl}
نحو و معنایی که برای توصیف چهارچوب پیشنهادی در این پژوهش مورد استفاده قرار گرفته است، در قالب یک مجموعه‌ی ابزار نیز پیاده‌سازی نیز شده 

\chapter{ارزیابی}
\label{chapter:evaluation}
\thispagestyle{plain}
در فصل قبل اجزای چهارچوب پیشنهادی این پژوهش به تفصیل تشریح شد و در م
\section{روش ارزیابی}
\section{ارزیابی کارایی}
سبیس
\section{ارزیابی تغییرپذیری}
سبیس
\subsection{بررسی معیارهای  ایستا }
با توجه به بزرگی سیستم مورد مطالعه، برای این مطالعه‌ی موردی دو مورد کاربرد از مجموعه‌ی مهم‌تری
\subsection{اعمال تغییرات}
\subsubsection{تغییر اول}

\section{نتایج ارزیابی}
قبل از بررسی نتایج، لازم است برخی نکات در مورد اجرای آزمون‌ها مورد بررسی قرار گیرد. مطابق آن‌چه در فصل قبل بیان شد، برای اجرای متوالی مجموعه‌های آزمون مختلف لازم است معیاری برای خاتمه‌ی هر مجموعه‌ی آزمون معرفی شود. با توجه به یکسان بودن احتمال بروز همه‌ی سناریوهای رفتاری در این مطالعه‌ی موردی و هم‌چنین تجربیات حاصل از چند اجرای آزمون‌ مقدماتی، در این مورد خاص از معیار زمانی برای خاتمه‌ی هر مجموعه و اجرای مجموعه‌ی بعدی استفاده شده است. اگر چه این معیار، ابتدایی‌ترین شرط خاتمه محسوب می‌شود، اما درعوض ساده‌ترین و کم هزینه‌ترین معیار نیز محسوب می‌شود و همان‌طور که اشاره شد در مورد جاری نتایج نسبتاً قابل قبولی نیز تولید می‌کند.

برای ارزیابی آزمون‌ها و مقایسه‌ی حالت‌های مختلف در این مطالعه‌ی موردی، از معیار پوشش متن\LTRfootnote{code coverage}  استفاده شده است. به این ترتیب مجموعه‌ی آزمونی بهتر فرض می‌شود که پوشش بالاتری از متن برنامه دارد. برای اندازه‌گیری این معیار نیز از ابزار متن باز \lr{Cobertura}\cite{cobertura} استفاده شده است. 

برای ارائه‌ی تحلیلی از کارایی روش معرفی شده در این پژوهش، نتایج آزمون‌های تولید شده توسط مدل‌های بخش قبل (پوشش متن حاصل از انجام آزمون‌های تولید شده بر مبنای این روش) باید با روش شناخته شده‌ی دیگری مقایسه شود. در این مطالعه‌ی موردی این نتایج با نتایج استفاده از روش اولیه‌ی آزمون مبتنی بر مدل مقایسه شده است. یادآوری این نکته ضروری است که الگوریتم آزمون مبتنی بر مدل استاندارد، امکان تبادل داده با سیستم را در طول آزمون ندارد. از آن‌جا که سیستم مورد این مطالعه به طور ذاتی مبتنی بر داده است، غنی کردن آزمون‌های رفتاری تولید شده از روش آزمون مبتنی بر مدل استاندارد، در آداپتور پیاده‌شده برای آن انجام شده است. به این معنی که مقادیر داده‌ای مربوطه پیش از اجرا در آداپتور نوشته شده و به ازای هر پیغام دریافت شده از آزمون‌گر، آداپتور با افزودن این مقادیر به پیغام، آن را کامل کرده و برای سیستم ارسال می‌کند. در مورد پاسخ‌های دریافتی از سیستم نیز مشابه همین روال برای پیغام‌های دریافت شده اتفاق می‌افتد. واضح است که با استفاده از این ایده فقط یک مقداردهی داده‌ای مورد آزمون قرار می‌گیرد.

با استفاده از این روش، آزمون‌های تولید شده بر پایه‌ی روش پیشنهادی این پژوهش با نتایج دو آزمون دیگر مقایسه می‌شوند: یکی آزمونی مبتنی بر آی‌اوکو استاندارد که داده‌های تعبیه شده در آن منجر به انجام کامل عملیات در سیستم (اجرای سناریوی اصلی مورد کاربرد) می‌شود؛ و دیگری آزمونی مبتنی بر آی‌اوکو استاندارد که داده‌های تعبیه شده در آن منجر به بروز خطا و اتمام ناقص عملیات در سیستم (اجرای یکی از سناریوهای فرعی مورد کاربرد) می‌شود. این کار برای هر سه مجموعه‌ی آزمون ذکر شده در بخش قبل انجام شده و نتایج در جدول \ref{table:testResults} ذکر شده است.


\subsection{تحلیل نتایج}
با داشتن نتای
\chapter{جمع‌بندی و نکات پایانی}
\label{chapter:Conclusion}
\thispagestyle{plain}
به عنوان جمع بندی متن حاضر، در این فصل به فهرستی از مهم‌ترین دستاوردهای این پژوهش خواهیم پرداخت. در مورد هر یک از این دستاوردها برخی نکات مهم نیز ذکر شده است. بعد از این، برخی از مهم‌ترین کاستی‌های چهارچوب ارائه شده آورده شده است. این کاستی‌ها در هر دو جنبه‌ی نظری و عملی مورد بررسی قرار گرفته‌اند. در نهایت، بر مبنای این موارد برخی جهت‌گیری‌های ممکن برای ادامه‌ی این پژوهش در آینده آورده شده است.

\section{دستاوردهای این پژوهش}
این پژوهش، چهارچوبی بدیع برای آزمون سیستم‌های نرم‌افزاری بر پایه‌ی روش‌های مبتنی بر مدل ارائه می‌کند که با استفاده از فرآورده‌هایی کاربردی که در فرآیند تولید نرم‌افزار تولید می‌شوند (مانند موارد کاربرد) و هم‌چنین دانش پیاده‌سازی نرم‌افزار برای تولید مدل‌های رفتاری و داده‌ای استفاده کرده، سپس به طور سیستماتیک به تولید خودکار موارد آزمون می‌پردازد. در تعریف این چهارچوب این امکان به وجود آمده که داده‌های آزمون نیز در مدلی هم‌خوان و در یک زبان یکسان با مدل‌های رفتاری (زبان یو‌ام‌ال)، طراحی شده و اطلاعات آن‌ها در مدل‌های رفتاری نیز مورد استفاده قرار گیرد. روش پیشنهاد شده‌ی این پژوهش، در ادامه برای تولید یک ابزار آزمون‌گر مورد استفاده قرار گرفته است. این ابزار از نظر نحوه‌ی تولید موارد آزمون، با توجه به ساختار الگوریتم \ref{algorithm:final}، به صورت \gls*{در-لحظه} عمل می‌کند. در فصل قبل نشان دادیم چگونه از این ابزار برای آزمون یک سیستم واقعی استفاده می‌شود.

در واقع چهارچوب پیشنهاد شده تلاش می‌کند تا مجموعه‌ی به هم‌پیوسته‌ای از فعالیت‌ها برای آزمون را، از اولین مراحل طراحی تا نتیجه‌گیری از مجموعه‌ی آزمون‌ها، پیشنهاد کند. در زیر برخی از مهم‌ترین دستاورد‌های هر یک از مراحل این کار آمده است:
\begin{strict_itemize}
\item برای طراحی مدل‌ها به طور مستقیم از فرآورده‌های تولید نرم‌افزار مانند مجموعه‌ی موارد کاربرد و یا استانداردهای موجود (مانند استاندارد \lr{ISO8583} در مطالعه‌ی موردی) استفاده شده است. هم‌چنین راهکارهایی برای تولید مستقیم مدل‌ها از موارد کاربرد (مثل قواعد مربوط به ایجاد یک ماشین برای هر مورد کاربرد و شکستن آن به اجزا) پیشنهاد گردیده است. به این ترتیب قدمی به سوی مدل‌سازی روش‌مند، به هدف آزمون برداشته شده و از اتکای کامل  به دانش ضمنی طراح آزمون پرهیز شده است.

\item تمامی توصیف‌ها، چه رفتاری و چه داده‌ای، در زبان یوام‌ال طراحی شده‌اند. علاوه بر تأثیرات مثبتی که استفاده از این زبان در طراحی آزمون‌گر داشت (مانند استفاده از ابزارهای طراحی قدرتمند و مفسّرها)، باید این را نیز افزود که امروزه حجم بسیار وسیعی از افرادی که در حوزه‌ی مهندسی نرم‌افزار فعالیت می‌کنند و حتی برخی از مشتریان سیستم‌های نرم‌افزاری به زبان یوام‌ال تسلط نسبی دارند و یا لااقل با آن آشنا هستند. این نکته خود باعث می‌شود که اولاً استفاده از راه‌کارهای ارائه شده در این پژوهش برای کاربران نهایی آسان شود و ثانیاً ارائه‌ی آن به مشتریان و کاربران خارجی نیز با سهولت بیشتری صورت گیرد.

\item الگوریتم آزمون مبتنی بر مدل به همراه داده که در \cite{FTW05} پیشنهاد شده بود از جهات مختلف بهبود داده شده است. یکی از این بهبودها امکان توصیف محدودیت‌های محیطی در کنار توصیف رفتار سیستم است. علاوه بر آن، امکان ایجاد توصیف‌های رفتاری در چند ماشین (چه در مورد توصیف سیستم و چه در مورد رفتار محیط) فراهم آمده است. اگرچه هر دوی این موارد در آزمون‌گرهای دیگری (مانند \lr{Uppaal TRON}) پیاده‌سازی شده اما هیچ یک بر روی سیستم‌های به همراه داده عمل نمی‌کنند. 

مورد قابل توجه دیگر افزوده شدن شرط خاتمه برای آزمون‌هاست (شرط $\xi$ در الگوریتم \ref{algorithm:enviocoFTCGen}). این شرط امکان این را فراهم می‌سازد تا در مورد اتمام مراحل آزمون تصمیم‌گیری شود. پیاده‌سازی این شرط نیز با معیارهای متفاوت و با توجه به نیاز صورت خواهد گرفت.

\item روشی برای مدل‌سازی داده‌های آزمون معرفی شده است که به طور کامل با الگوریتم تولید موارد آزمون هم‌خوان می‌باشد. این روش با استفاده از روش افراز رده‌ای و با اتکا به فرآورده‌های تولید نرم‌افزار و هم‌چنین دانش حاصل از متن برنامه، می‌تواند مقادیر داده‌ای را به همراه ارتباطات و محدودیت‌های بین آن‌ها مدل‌سازی نماید. استفاده از زبان مشترک یو‌ام‌ال برای توصیف این موارد تفاوت ماهوی این مدل‌ها را با مدل‌های رفتاری تا حد خوبی پوشش می‌دهد.

\item مفهوم سناریوهای آزمون، به هدف مشخص کردن مجموعه‌های آزمون که سیستم را در حضور محدودیت‌های محیطی مختلف مورد آزمون قرار می‌دهند، معرفی گردید. به علاوه، این سناریوها وظیفه‌ی مشخص نمودن مقادیر اولیه‌ی داده‌ای برای سیستم را به ازای هر مجموعه‌ی آزمون بر عهده گرفتند. این مسئله از این بابت حائز اهمیت است که الگوریتم ارائه شده در \cite{FTW05} در مورد تعیین مقادیر اولیه ($\iota$) برای ماشین حالت نمادین، چه در معناشناسی و چه در الگوریتم آزمون، اظهار نظر نمی‌کند اما ترکیب این مقادیر ممکن است بر رفتار سیستم اثری جدی داشته باشد. به این ترتیب استفاده از سناریو‌های آزمون می‌توانند راه‌کاری برای این مسئله باشد. در واقع سناریوهای آزمون را می‌توان به عنوان حلقه‌ی اتصال مدل‌های ساختار داده‌ای (که با روش‌هایی مانند افراز رده‌ای شکل می‌گیرند) و مدل‌های رفتاری (که از توصیف‌های رفتاری سیستم مثل موارد آزمون ساخته می‌شوند) دانست.

\item آزمون‌گری با پشتیبانی از ایده‌های ذکر شده طراحی و تولید شده است. این آزمون‌گر از توصیف‌های استاندارد \lr{XMI} (که یک استاندارد مبتنی بر \lr{XML} برای توصیف نمودارهای یوام‌ال است) بهره گرفته و موارد آزمون  مورد نظر را تولید می‌کند. در نهایت، نتیجه‌ی آزمون در قالب یک رأی اعلام می‌شود. با توجه به ضعف پشتیبانی ابزاری از روش آزمون مبتنی بر مدل به همراه داده، طراحی این آزمون‌گر از اهمیت ویژه‌ای برخوردار است.

\item کارایی روش پیشنهاد شده با مطالعه‌ی موردی یک سیستم واقعی، سنجیده شده است. اگرچه کارایی این روش را نمی‌توان تنها با استناد به یک مطالعه‌ی موردی نمایش داد،‌ با این حال استفاده از چهارچوب پیشنهادی و ابزارهای آن، تجربیات بسیار ارزنده‌ای در پی داشت که در برخی از مهم‌ترین ‌آن‌ها در فصل قبل بررسی شد. علاوه بر این در انتخاب مورد مطالعه تلاش شد تا نرم‌افزاری جامع در حوزه‌ی نرم‌افزارهای مالی (که خود گستره‌ی بزرگی را شامل می‌شود) انتخاب شود تا به این ترتیب بتوان به طور تقریبی از قابل استفاده بودن این روش در نمونه‌های دیگر نیز اطمینان حاصل نمود.
\end{strict_itemize}

\section{کاستی‌های چهارچوب}
چهارچوب پیشنهاد شده در این پژوهش دارای کاستی‌هایی نیز هست که کار بیشتری را می‌طلبد. در این بخش به طور فهرست‌وار به برخی از آن‌ها اشاره می‌کنیم:
\begin{itemize}
\item اگرچه تعریف \ref{definition:envdataioco} پایه‌ای نظری برای روال تولید موارد آزمون در حضور محدودیت‌های محیطی بنا می‌کند و در ادامه الگوریتم \ref{algorithm:final} روال تولید موارد آزمون را شرح می‌دهد، با این حال در این مرحله مقایسه‌ای بین قدرت بیان این تعریف از آی‌اوکو با تعریف اصلی آی‌اوکو به همراه داده (تعریف \ref{definition:dataioco}) انجام نمی‌شود. این مقایسه لازم است به طور نظری نشان دهد که آیا این دو تعریف، توصیف‌کننده‌ی یک رابطه‌ی مطابقت هستند و یا خیر و در صورت معادل نبودن در مورد جزئیات ارتباط این دو نیز بحث شود.

\item در این متن شرط خاتمه‌ی مجموعه‌ی آزمون ($\xi$) به طور کلی تعریف و در مورد برخی از معیارهای تعریف آن نیز بررسی اجمالی انجام شد. با این حال جزئیات تعریف هر یک از این معیارها، مقایسه‌ی آن‌ها از دید معناشناسی با یکدیگر و هم‌چنین روش پیاده‌سازی آن‌ها در این متن پوشانده نشده است. این صورت مسئله خود نیاز به پژوهش‌های مستقلی دارد چنان‌چه برخی از این معیارها هم‌اکنون در قالب پژوهش‌های دیگری در حال توسعه می‌باشند (برای مثال در \cite{Briones06} چهارچوبی برای اندازه‌گیری پوشش مدل ارائه شده است).

\item چهارچوب پیشنهاد شده در این متن تنها قادر است به صورت \gls*{در-لحظه} موارد آزمون را تولید کند. این روش اگرچه مزایای بسیاری دارد اما لزوماً در همه‌ی موارد بهترین گزینه نیست. برای مثال در حالت‌هایی که نیاز به \glspl*{آزمون بازگشتی}\LTRfootnote{regression tests} وجود دارد تولید هرباره‌ی حجم زیادی از موارد آزمون به صرفه نیست. در این موارد روش‌های تولید آزمون به صورت \gls*{برون‌خط} به همراه یک روش برای \gls*{اولویت‌بندی}\LTRfootnote{prioritization} موارد آزمون می‌تواند کارساز باشد. 

\item آزمون‌گر تولید شده، در حال حاضر از قابلیت \gls*{بررسی گونه‌ها}\LTRfootnote{type checking} برای گونه‌های داده‌ای طراحی شده پشتیبانی نمی‌کند. این امر سبب می‌شود تا امکان بروز خطای انسانی در طراحی مقادیر داده‌ای زیاد باشد. این مسئله با توجه به تنوع مقادیر مختلف داده‌ای که می‌تواند مورد استفاده قرار گیرد، بیشتر خودنمایی می‌کند.

\item به جز پشتیبانی از داده‌های آزمون، گسترش‌های دیگری از رابطه‌ی مطابقت آی‌اوکو وجود دارد. برای مثال آزمون سیستم‌های زمان‌دار توسط رابطه‌ی مطابقت $\mathbf{rtioco}$ توصیف شده و در ابزار \lr{Uppaal TRON} نیز پیاده‌ شده است \cite{Larsen04Tron}. برای تکمیل آزمون‌گر پیاده‌شده در این پژوهش لازم است پشتیبانی از این گسترش‌ها نیز به آن افزوده شود.

\item تبدیل انجام شده از زبان یوام‌ال به ماشین‌های گذار نمادین هنوز نادقیق است. دلیل این امر، گستردگی بسیار زیاد زبان یو‌ام‌ال و نیاز به تعریف نحو و معنای هر یک از اجزای آن در آزمون‌گر طراحی شده است. در این پژوهش، به طور ضمنی زیرمجموعه‌ی پشتیبانی شده از زبان یو‌ام‌ال معادل عناصری در نظر گرفته شده است که نحو و معنای آن‌ها در این متن تشریح شد.
\end{itemize}
 
\section{جهت‌گیری‌های پژوهشی آینده}
پژوهش حاضر، اگرچه ایده‌ای منسجم را در حوزه‌ی آزمون‌های مبتنی بر مدل دنبال می‌کرد، اما می‌توان آن را به عنوان نقطه‌ی آغازی برای چندین پژوهش مرتبط دانست. برخی ایده‌های مطرح شده در این متن هنوز ابتدایی هستند و برخی نیازها نیز با روش پیشنهادی قابلیت مدل‌سازی را ندارند. اگرچه تکمیل و رفع کاستی‌های ذکر شده در بالا را می‌توان به عنوان کارهای آینده در راستای این پژوهش دانست، اما علاوه بر آن‌ها برخی از جهت‌گیری‌های احتمالی برای پژوهش‌های آینده که از ایده‌ی این پژوهش بهره می‌برند را نیز می‌توان به شکل زیر برشمرد:
\begin{itemize}
\item ایده‌ی پیشنهادی در این پژوهش، راه را برای ایجاد یک \emph{فرآیند آزمون نرم‌افزار}\LTRfootnote{software testing process} کامل و جامع هموار می‌سازد. چنین فرآیندی لازم است در کنار فرآیند تولید نرم‌افزار قابل استفاده بوده و در آن ارتباط این دو نیز فرآیند نیز تعریف شود.

\item ایده‌ی افراز رده‌ای اگرچه چهارچوبی سطح بالا برای مدل‌سازی داده‌های آزمون و ارتباطات آن‌ها فراهم می‌کند. با این‌حال تولید مقادیر داده‌ای باید در این روش به صورت کاملاً دستی انجام شود. در ادامه لازم است روش‌های تولید رده‌ها و مقادیر داده‌ای به صورت نیمه‌خودکار و یا خودکار مورد نظر قرار گیرد.

\item بررسی مطالعه‌ی موردی انجام شده در این پژوهش، این ایده را تقویت می‌کند که ممکن است بتوان با استفاده از اطلاعات خاص دامنه‌ی سیستم (در مطالعه‌ی این پژوهش دامنه‌ی سیستم‌های مالی)، به روشی برای تولید توصیف‌های دقیق‌تر و موارد آزمون با کیفیت بالا دست یافت. بنابراین یک مسئله‌ی مهم که می‌تواند به عنوان جهت‌گیری احتمالی این پژوهش مورد نظر قرار گیرد، طراحی روش‌های آزمون و ابزارهایی است که به صورت \emph{\gls*{خاص-دامنه}}\LTRfootnote{domain specific} بهینه شده باشند.
\end{itemize}



%\appendix
%\chapter{تطبیق نمادگذاری‌ها}
%\label{appendix}
%\thispagestyle{plain}
%\section*{متن برنامه‌ی طراحی شده به روش ارسال  ناهمگام پیغام}
سلام
\section*{متن برنامه‌ی طراحی شده به روش شیءگرا}
تعریف ذکر شین گذار نمادین به طور ساده به این شکل است:


\newpage

\linespread{1.2}

\small{
\bibliographystyle{ieeetr-fa}
\clearpage
\phantomsection
\addcontentsline{toc}{chapter}{کتاب‌نامه}
\bibliography{references}
}

\newpage
\begin{multicols}{2}
\glossarystyle{mylist} 
\def\glossaryname{واژه‌نامه‌ی فارسی به انگلیسی}
\printglossary
\addcontentsline{toc}{chapter}{واژه‌نامه‌ی فارسی به انگلیسی}
\end{multicols}


\begin{latin}
\pagestyle{empty}


% LATIN ABSTRACT
\newpage
{\centering\Large{\bf{Design of  Domain Logic Using Asynchronous Message Passing }} \par}
{\centering\small{\bf{Abstract}} \par \vskip 1cm}
\noindent In recent years, interest in Actor model has been growing, among researchers as well as practitioners. This interest is triggered by emerging programming platforms such as multicore computers and cloud computers. In some cases, such as cloud computing, the Actor model is a natural programming model because of the distributed nature of these platforms. This trend in using concurrent programming using actors, makes the need for providing design principles and patterns in this model just like they are provided thoroughly in sequential  object-oriented design books.
In this research, we choose a simple domain model named simple educational system and take the design steps needed to implement it using asynchronous message passing. The extracted patterns of actor interactions and messaging styles are provided to be used in simillar design attempts. Moreover, an empirical evaluation of software quality metrics for the design is undertaken and the results are compared with a sequential oop approach for the same domain model.
{\par\vspace{5mm}}
\noindent\textbf{Keywords: }\textit{asynchronous message passing, design patterns, object-oriented design, domain modeling}
% END OF LATIN ABSTRACT

\newpage
\mbox{}


%%%%%%%%%%%%%%%%%%%%%%%%%%%%%
% LATIN TITLE PAGE
%%%%%%%%%%%%%%%%%%%%%%%%%%%%%
\font\titlefont=cmssbx10 scaled 2074
\font\supervisorfont=cmbxti10
\newpage
\thispagestyle{empty}
\begin{center}
\begin{tabular}{lp{7cm}r}
\includegraphics[width=3.8cm]{Figures/englogo} & & \includegraphics[width=2.8cm]{Figures/utlogo} \\
\end{tabular}

\vskip 1cm
\large\bfseries
University of Tehran \par
School of \space Electrical and Compuer Engineering
\par
\vskip 1.5cm
\addtolength{\baselineskip}{5mm}
{\titlefont Design of  Domain Logic Using Asynchronous Message Passing} \par
\addtolength{\baselineskip}{-5mm}
\vskip 1cm
{\bfseries by}\par
{\Large\bfseries Vahid Zoghi}\par
\vskip 1cm
Under supervision of \\
{\supervisorfont\Large Dr. Ramtin Khosravi}
\par
\vskip 2cm
{A thesis submitted to the Graduate Studies Office \\ in partial fulfillment of the requirements \\ for the degree of M.Sc \par
in
\par
\large Computer Engineering}
\par
\vskip 1cm
{Sep 2012}
\par
\vfill
\end{center}
%%%%%%%%%%%%%%%%%%%%%%%%%%%%%
% END OF LATIN TITLE

\end{latin}
\end{document}
